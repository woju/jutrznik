%
% (c) 2016-2018 Wojtek Porczyk <w.porczyk@warszawa.oaza.pl>
% Licence: GPL-3 or later. No warranty.
%

\unprotect

% fonty {{{1
\definefontfeature[default][default][protrusion=quality,expansion=quality,script=latn]
\enabletrackers[fonts.missing]
\starttypescript[sans][futura]
    % niestety, ciężko znaleźć dobrą futurę na githubie, więc zadowolimy się
    % amerykańską podróbką futury, spartanem
    \definefontsynonym[Futura-Bold][file:fonts/LeagueSpartan-Bold.otf]
\stoptypescript
\starttypescript[sans][futura][name]
    \setups[font:fallback:sans]
    \definefontsynonym[SansBold][Futura-Bold][features=default]
    \definefontsynonym[SansCaps][Futura-Bold][features=smallcaps]
\stoptypescript
\starttypescript[jutrznik]
    \definetypeface[jutrznik][rm][serif][libertine][default]
    \definetypeface[jutrznik][ss][sans] [futura]   [default]
%   \definetypeface[jutrznik][tt][mono] [iosevka]  [default]
%   \definetypeface[jutrznik][mm][math] [junicode] [default]
    \definefontsynonym[Serif][file:fonts/LinLibertine_R.otf]
\stoptypescript

% varia {{{1
\directlua{dofile('breviarium.lua')}

\setupdelimitedtext[quote:1][left=„,right=”]
\setupdelimitedtext[quote:2][left=«,right=»]
\setupdelimitedtext[quotation:1][left=«,right=»]
\setupdelimitedtext[quotation:2][left=„,right=”]

\def\rubrum#1{%
    \doifmodeelse{bw}{#1}{\color[red]{#1}}%
}
\def\flexa{\rubrum{†}}
\def\flxs{\flexa\space}
\def\flexas{\flxs}
\def\mediatio{\rubrum{*}}
\def\meds{\mediatio\space}
\def\mediatios{\meds}

\doifmodeelse{bw}{%
    \definestartstop[rubrica][style=\tfx\bf,before={\blank[small]},after={\blank[small]}]%
}{%
    \definestartstop[rubrica][style=\tfx,color=red,before={\blank[small]},after={\blank[small]}]%
}

\defineitemgroup[dialog]
\setupitemgroup[dialog][1][9][
    color=red,
    stopper=,
    style=,
    inbetween=,
    after=,
]

\setupparagraphnumbering[color=red]

% nagłówki i spis treści{{{1

% NUMBERED          UNNUMBERED
% part              -
%                       booktitle
% chapter           title
%     day
% section           subject
%                       hour
% subsection        subsubject
%                       hourpart
% subsubsection     subsubsubject
% ...               ...

% całość: booktitle
%   dodatek: booktitle
% dni: chapter
%   załączniki: title
% godzina: subject
% część godziny: subsubject

\definehead[booktitle][part]
\setuphead[booktitle][
    number=no]

\setuphead[chapter,title][
    style=\tfb\ss\bf,
    align=middle,
    header=empty,
]
\definehead[specialtitle][title]
\definehead[day][chapter]
\setuphead[day][
    sectionstarter={Dzień\space},
    sectionstopper=:,
    sectionsegments=day,
]

\setuphead[subject][
    style=\bf,
    deeptextcommand=\WORD,
    align=middle,
    page=preference,
]
\definehead[hour][subject]
\setuphead[hour][
    deeptextcommand=\WORD,
    after=\blank,
]

\setuphead[subsubject][
    style={\sc},
    color=red,
    page=no,
    before={\blank[line]},
    after={\blank[small]},
]
\definehead[hourpart][subsubject]
\def\hourpartx#1#2{\rubrum{\sc #1} #2}

\setuphead[subsubsubject][
    color=red,
    style=,
    align=middle,
    page=no,
    before=\blank,
    after=\blank,
]

\def\datesubtitle#1{%
    \blank[back]\startalignment[center]\rubrum{#1}\stopalignment}

% spis treści

\setupcombinedlist[content][
    list={booktitle,day,title,specialtitle},
    alternative=c,
    criterium=all,
]
\setuplist[interaction=all,color=black,pagecolor=red]
\setuphead[booktitle,title,specialtitle][
    incrementnumber=list]
\setuphead[chapter][
    incrementnumber=yes]
%\setuphead[booktitle,title][
%    number=no]

\setuplist[day][margin=5mm,width=15mm,before=]
\setuplist[title][headnumber=no,margin=5mm]
\setuplist[specialtitle,booktitle][
    margin=0mm,
    before={\blank[2.5mm]},
    after={\blank[2.5mm]},
]
\setuplist[day,title][before={\blank[1mm]}]

% hymny {{{1
\definelines[hymnusn][
    before=,
    after=,
    option=packed,
]
\defineitemgroup[hymnus]
\setupitemgroup[hymnus][1][n][
    color=red,
    stopper=,
    before={\starthymnusn},
    after={\stophymnusn},
    inbetween={\blank[small]},
]

% psalmy {{{1
\newcount\psalmusverse
\definehead[psalmus][subsubsubject]
\setuphead[psalmus][
    beforesection={\startlines\psalmusverse=0\relax\startnarrower[0mm]},
    aftersection={%
Chwała Ojcu i~Synowi,\pmed\par
i~Duchowi Świętemu.\pfin\par
Jak była na początku, teraz i~zawsze,\pmed\par
i~na wieki wieków. Amen.\pfin\stopnarrower\stoplines},
]
\definehead[psalmusx][psalmus]
\setuphead[psalmusx][beforesection=,aftersection=]

% \unexpanded is needed for finding in antiphona, so it can be stripped in \antr
\unexpanded\def\cont{\rubrum{†}}
\def\pflx{~†\page[no]}
\def\pmed{~*\page[no]}
\def\pfin{%
    \stopnarrower
    \advance\psalmusverse 1%
    \startnarrower[%
        \ifodd\psalmusverse
            5mm%
        \else
            0mm%
        \fi
    ]%
}


% antyfona {{{1

% XXX ostrożnie z tym, łatwo zepsuć

\begingroup
    \catcode`\*=\active
    \catcode`\/=\active
    \gdef* {\color[red]{\string*} }%
    \gdef/ {\color[red]{\string/} }%
\endgroup

\def\ant{%
    \dodoubleempty\!ant
}

\def\!ant[#1][#2]#3\par{%
    \blank
    \ctxlua{userdata.ant.doant(\!!bs#1\!!es, \!!bs#2\!!es, \!!bs#3\!!es)}
}
\def\antr{
    \ctxlua{userdata.ant.doantr()}%
}

% prośby {{{1

\definesymbol[9][\textormathchar{"2015}]
\defineitemgroup[prosbaig]
\setupitemgroup[prosbaig][1][9,joinedup][color=red]

\def\aklamacja#1\par{%
    \ctxlua{buffers.assign('aklamacja', \!!bs#1\!!es)}%
    {\it #1}%
    \par
    \blank[2.5mm]}

\def\prosba#1---#2\par{%
    \vbox{%
        #1\par
        \startprosbaig[]
        \item\ignorespaces#2\par
        \stopprosbaig
        {\it \getbuffer[aklamacja]}%
    }%
    \blank[2.5mm]%
    \par
}

% }}}

\protect
% vim: fdm=marker ts=4 sts=4 sw=4 et

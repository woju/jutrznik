\enablemode[preprint]

% {{{1 prefacja
\usemodule[breviarium]

\enableregime[utf]
\setuplanguage[pl][date={year,-,mm,-,dd}]
\mainlanguage[pl]
\enabletrackers[fonts.missing]
\setupinteraction[
    title={Jutrznia w Oazie Nowego Życia 2°},
    author={Diecezjalna Diakonia Liturgiczna Archidiecezji Warszawskiej},
    state=start,
    color=red,
    style=,
]

\setupbackend[
    format={pdf/a-1a:2005},
    profile={default_cmyk.icc,default_rgb.icc,default_gray.icc},
    intent={ISO coated v2 300\letterpercent\space (ECI)},
]

\setupstructure[state=start,method=auto]
\placebookmarks[booktitle,day,title,specialtitle][force=yes]

\doifmodeelse{preprint}{
    \setuppapersize[A5]
}{
    \setuppapersize[A5][A4,landscape]
    \setuparranging[2UP]
}
\setuplayout[
    location={middle,middle},
    height=190mm,
    topspace=10mm,
    header=7mm,
    footer=7mm,
    headerdistance=0mm,
    footerdistance=0mm,
]
%\showframe
\setupbodyfont[jutrznik,10.5pt]
\setuppagenumbering[location=footer,style={\switchtobodyfont[8pt]\futura}]
\startsetups[header]
\framed[frame=off,bottomframe=on,width=\makeupwidth]{%
    \tfx\rubrum{\getmarking[daynumber]~\getmarking[day]}%
}
\stopsetups
\startsetups[header2]
\framed[frame=off,bottomframe=on,width=\makeupwidth]{%
    \tfx\rubrum{\getmarking[title]}%
}
\stopsetups
\setupheadertexts[\setups{header}]

%}}}1

\starttext

% strona tytułowa %{{{1
\setupheader[state=high]
\setupfooter[state=high]
\startbooktitle[title={Jutrznia},list={Strona tytułowa}]
\dontleavehmode\vfill

\startalignment[middle]
\switchtobodyfont[48pt]\apropal Jutrznia
\blank[40mm]
\switchtobodyfont[20pt]\apropal w~Oazie Nowego Życia 2°

\vfill
\switchtobodyfont[10.5pt]\futura
\rubrum{\uppercase{wersja robocza \currentdate}}\blank[10mm]
\uppercase{%
Diecezjalna Diakonia Liturgiczna\crlf
Archidiecezji Warszawskiej

\blank[5mm]
Warszawa 2018}
\stopalignment

% Dzień  1: Niewola {{{1
\page[even]
\setupheader[state=high]
\setupfooter[state=high]
\externalfigure[d01-wzniesione-rece][width=\textwidth]
\startday[title={Niewola}]

\starthour[title={Wezwanie}] %{{{2
\input lib/panie-otworz

\ant Uwielbiajmy Pana i~Króla,~/ który do nas przyjdzie.

\input lib/invit-ps095

\input lib/zapalenie-A

\starthour[title={Jutrznia}] %{{{2

\starthourpart[title={Hymn}]
% http://brewiarz.pl/indeksy/pokaz.php3?id=3&nr=076
% Adwent (przed 17 grudnia)
% LG tom I, wydanie I (Pallottinum 1982), s. 128
% LG tom I, wydanie II (Pallottinum 2006), s. 142
% Skrócona LG (Pallottinum 1991), s. 44
\starthymnus
\item Otwórz się, niebo pokryte chmurami,
I~Sprawiedliwy niech zejdzie z~obłoków
Jak deszcz ożywczy, co zwilży pustynię
Każdego serca.
%
\item Otwórz się, ziemio, i~wydaj swój Owoc
Zrodzony z~Ojca i~czystej Dziewicy,
Gdyż On przyniesie zbawienie i~pokój
Całemu światu.
%
\item Przyjdź, Pojednanie, już więcej nie zwlekaj;
Niech Twa obecność przywróci nam łaskę,
I~wprowadź wszystkich znużonych wędrowców
Do domu Ojca.
%
\item Znak Twój na niebie zabłyśnie narodom
Gdy znów nadejdziesz, by sądzić sumienia;
Pamiętaj wtedy, żeś mieszkał wcielony
Pomiędzy nami.
%
\item Boży Baranku, czekamy na Ciebie
I~pieśnią chwały wielbimy Twą dobroć;
Sławimy również i~Ojca, i~Ducha
W~jedności z~Tobą. Amen.\stophymnus

\starthourpart[title={Psalmodia}]
\ant[n=1] Kiedy przyjdę~* i~ujrzę oblicze Boże?

\startpsalmus[title={Psalm 42}]
Jak łania pragnie wody ze strumieni,\pmed
tak dusza moja pragnie Ciebie, Boże.\pfin
Dusza moja Boga pragnie, Boga żywego,\pmed
kiedyż więc przyjdę i~ujrzę oblicze Boże?\pfin
Łzy są moim chlebem we dnie i~w~nocy;\pmed
„Gdzie jest twój Bóg?” pytają mnie co dzień.\pfin
Rozpływa się we mnie moja dusza,\pmed
gdy wspominam, jak z~tłumem kroczyłem do Bożego domu\pfin
W~świątecznym orszaku,\pmed
wśród głosów radości i~chwały.\pfin
Czemu zgnębiona jesteś, duszo moja,\pmed
i~czemu trwożysz się we mnie?\pfin
Ufaj Bogu, bo jeszcze wysławiać Go będę:\pmed
On zbawieniem mojego oblicza i~moim Bogiem!\pfin
A~we mnie samym dusza przygnębiona,\pflx
przeto wspominam Cię z~ziemi Jordanu,\pmed
z~ziemi Hermonu i~góry Misar.\pfin
Głębia przyzywa głębię hukiem wodospadów.\pmed
Wszystkie Twe nurty i~fale nade mną się przewalają.\pfin
Niech Pan udzieli mi we dnie swej łaski,\pflx
a~w~nocy będę Mu śpiewał,\pmed
będę sławił Boga mego życia.\pfin
Mówię do Boga: Opoko moja, czemu zapominasz o~mnie?\pmed
Czemu chodzę smutny, gnębiony przez wroga?\pfin
Kości we mnie się kruszą,\pmed
gdy lżą mnie przeciwnicy,\pfin
Gdy cały dzień mówią do mnie:\pmed
„Gdzie jest Bóg twój?”\pfin
Czemu zgnębiona jesteś, duszo moja,\pmed
i~czemu trwożysz się we mnie?\pfin
Ufaj Bogu, bo jeszcze wysławiać Go będę:\pmed
On zbawieniem mojego oblicza i~moim Bogiem!\pfin
\stoppsalmus

\antr

\ant[n=2] Okaż nam, Panie,~* miłosierdzie swoje.

\startpsalmus[title={Pieśń (Syr 36,1-5.10-13)}]
Zmiłuj się nad nami, Panie, Boże wszystkich rzeczy,\pmed
i~spójrz i~ześlij bojaźń przed Tobą na wszystkie narody.\pfin
Wyciągnij rękę przeciw obcym ludom,\pmed
aby ujrzały Twoją potęgę.\pfin
Bo jak przez nas okazałeś im świętość swoją,\pmed
tak przez nich wobec nas okaż się wielkim.\pfin
Niech i~one uznają, jak my uznajemy,\pmed
że nie ma Boga, oprócz Ciebie, Panie.\pfin
Powtórz znaki i~znów uczyń cuda,\pmed
wsław swoją rękę i~ramię prawe.\pfin
Zgromadź wszystkie pokolenia Jakuba\pmed
i~weź je w~posiadanie, jak było na początku.\pfin
Panie, zlituj się nad narodem nazwanym Twoim imieniem,\pmed
nad Izraelem, którego uznałeś za pierworodnego.\pfin
Miej miłosierdzie nad Twym świętym miastem,\pmed
nad Jeruzalem, miejscem Twego odpoczynku.\pfin
Napełnij Syjon wysławianiem Twej mocy,\pmed
a~Twój lud swoją chwałą.\pfin
\stoppsalmus

\antr

\ant[n=3] Błogosławiony jesteś, Panie,~* na sklepieniu nieba.

\startpsalmus[title={Psalm 19 A, 2-7}]
Niebiosa głoszą chwałę Boga,\pmed
dzieło rąk Jego obwieszcza nieboskłon.\pfin
Dzień opowiada dniowi,\pmed
noc nocy wiadomość przekazuje.\pfin
Nie są to słowa ani nie jest to mowa,\pmed
których by dźwięku nie usłyszano.\pfin
Ich głos się rozchodzi po całej ziemi,\pmed
ich słowa aż po krańce świata.\pfin
Tam słońcu namiot postawił,\pflx
a~ono jak oblubieniec wychodzi ze swej komnaty,\pmed
cieszy się jak siłacz ruszający do biegu.\pfin
Ono wschodzi na krańcu nieba\pflx
i~biegnie aż po drugi kraniec,\pmed
a~nic przed jego żarem się nie schroni.\pfin
\stoppsalmus

\antr

\starthourpart[title={Czytanie\hfill\tf Dz 7,17-29a}]
W~miarę, jak zbliżał się czas obietnicy, którą Bóg dał Abrahamowi, rozrastał
się lud i~rozmnażał w~Egipcie, aż doszedł do władzy inny król w~Egipcie, który
nic nie wiedział o~Józefie. Działał on podstępnie przeciwko naszemu narodowi
i~przymuszał ojców naszych do wyrzucania niemowląt, aby nie zostawały przy
życiu. Wówczas właśnie narodził się Mojżesz. Był on miły Bogu. Przez trzy
miesiące karmiono go w~domu ojca. A~gdy go wyrzucono, zabrała go córka faraona
i~przybrała go sobie za syna. Mojżesza wykształcono we wszystkich naukach
egipskich, i~potężny był w~słowie i~czynie. Gdy skończył lat czterdzieści,
przyszło mu na myśl odwiedzić swych braci, synów Izraela. I~zobaczył jednego,
któremu wyrządzono krzywdę. Stanął w~jego obronie i~zabiwszy Egipcjanina
pomścił skrzywdzonego. Sądził, że bracia jego zrozumieją iż Bóg przez jego ręce
daje im wybawienie, lecz oni nie zrozumieli. Następnego dnia zjawił się wśród
nich, kiedy bili się między sobą, i~usiłował ich pogodzić. Ludzie, braćmi
jesteście ---~zawołał ---~czemuż krzywdzicie jeden drugiego? Ten jednak, który
krzywdził bliźniego, odepchnął go. Któż ciebie ustanowił panem i~sędzią nad
nami? ---~zawołał ---~czy chcesz mnie zabić, tak jak wczoraj zabiłeś
Egipcjanina? Na te słowa Mojżesz uciekł i~żył jako cudzoziemiec w~ziemi Madian.

\starthourpart[title={Pieśń Zachariasza}]

\ant[title={Ant. do pieśni Zachariasza}] Bóg przez wielką swą miłość, jaką nas
ukochał,~* zesłał swego Syna w~ciele podobnym do ciała grzesznego.

\input lib/benedictus

\antr

\starthourpart[title={Prośby}]

Obchodząc uroczystość Zwiastowania, czcimy początek dzieła naszego
zbawienia. Z~radością więc skierujmy do Boga wspólne błagania:

\aklamacja Niech Bogarodzica wstawia się za nami.

\prosba Spraw, Boże, abyśmy z radością przyjęli naszego Zbawiciela,
--- jak Maryja Dziewica ochoczo przyjęła nowinę zwiastowaną Jej przez anioła.

\prosba O~dobry Ojcze, pamiętaj o~nas i~o~wszystkich ludziach,
--- jak wejrzałeś na pokorę swojej służebnicy.

\prosba Spraw, abyśmy zawsze zgadzali się z~Twoją wolą,
--- jak Maryja, nowa Ewa, posłusznie przyjęła Twoje Boskie Słowo.

\prosba Niechaj Maryja wspomaga ubogich, podtrzymuje małodusznych, pociesza
płaczących,
--- niech modli się za lud, oręduje za duchowieństwem, wstawia się za
poświęconymi Bogu kobietami.

Ojcze nasz.

\starthourpart[title={Modlitwa}]
Wszechmogący Boże, od dawna przygniata nas jarzmo grzechów,~\meds spraw, aby nas
wyzwoliło upragnione nowe narodzenie Twojego Syna. Który z~Tobą żyje i~króluje
w~jedności Ducha Świętego,~\meds Bóg, przez wszystkie wieki wieków.

\starthourpart[title={Zakończenie \tf (piosenka dnia)}]

% Dzień  2: Bóg Wybawiciel {{{1
\page[even]
\setupheader[state=high]
\setupfooter[state=high]
\externalfigure[d02-ihs][width=\textwidth]
\startday[title={Bóg Wybawiciel}]

\starthour[title={Wezwanie}] %{{{2
\input lib/panie-otworz

\ant Przyjdźcie, uwielbiajmy Pana,~* który jest Pasterzem swojego ludu
\startrubrica
Psalm Wezwania nr FIXME \at{str.}[invit-ps095].
\stoprubrica

\input lib/zapalenie-A

\starthour[title={Jutrznia}] %{{{2

\starthourpart[title={Hymn}]
% http://brewiarz.pl/indeksy/pokaz.php3?id=3&nr=083
% Jutrznia: Adwent (po 17 grudnia)
% LG tom I, wydanie I (Pallottinum 1982), s. 284
% LG tom I, wydanie II (Pallottinum 2006), s. 291
% Skrócona LG (Pallottinum 1991), s. 107
\starthymnus
\item Prorocy pełni natchnienia
Zwiastują Bożą nowinę:
Już wkrótce Chrystus nadejdzie,
Zbawienie niosąc ludzkości.
%
\item Dlatego światło poranka
Przenika serca weselem,
A~przyszłej chwały zapowiedź
Już dźwięczy w~hymnie radosnym.
%
\item Gdy pierwszy raz Jezus przyszedł,
To nie dla sądu nad światem,
Lecz aby rany uleczyć
I~uratować ginących.
%
\item A~drugie przyjście ostrzega,
Że Pan do bramy kołacze;
Nagrodzi wtedy wybranych
Wiecznością swego królestwa.
%
\item Szukamy Ciebie, o~Chryste,
By ujrzeć Twoje oblicze,
A~mając udział w~Twym szczęściu,
Wysławiać Trójcę na wieki. Amen.\stophymnus

\starthourpart[title={Psalmodia}]

\ant[n=1] Boże, Twoja droga jest święta,~* nikt nie dorówna wielkością naszemu
Bogu.

\startpsalmus[title={Psalm 77}]
Głos mój się wznosi do Boga, gdy wołam,\pmed
głos mój wznoszę do Boga, aby mnie usłyszał.\pfin
W~dniu mej niedoli szukam Pana,\pmed
w~nocy niestrudzenie wyciągam rękę.\pfin
Dusza moja jest niepocieszona,\pflx
jęczę, kiedy wspomnę Boga,\pmed
słabnie mój duch, gdy rozmyślam.\pfin
Ty spędzasz sen z~moich powiek,\pmed
z~niepokoju mówić nie potrafię.\pfin
Rozpamiętuję dni, które dawno minęły,\pmed
i~lata poprzednie wspominam.\pfin
Rozmyślam nocą w~swym sercu,\pmed
rozważam, a~duch mój docieka:\pfin
„Czy Bóg odrzuci na wieki\pmed
i~już nie będzie łaskawy?\pfin
Czy Jego łaskawość ustała na zawsze\pmed
i~słowo zamilkło na pokolenia?\pfin
Czy Bóg zapomniał o~litości,\pmed
czy w~gniewie powstrzymał swe miłosierdzie?”\pfin
I~mówię: „Jakże to bolesne,\pmed
że odwróciła się ode mnie prawica Najwyższego”.\pfin
Wspominam dzieła Pana,\pmed
oto wspominam Twoje dawne cuda.\pfin
Rozmyślam o~wszystkich Twych dziełach\pmed
i~czyny Twoje wspominam.\pfin
Boże, Twoja droga jest święta,\pmed
który z~bogów dorówna wielkością naszemu Bogu?\pfin
Ty jesteś Bogiem działającym cuda,\pmed
ludziom objawiłeś swą potęgę.\pfin
Ramieniem swoim Twój lud wybawiłeś,\pmed
synów Jakuba i~Józefa.\pfin
Boże, ujrzały Cię wody,\pflx
ujrzały Cię wody i~zadrżały,\pmed
wzburzyły się ich odmęty.\pfin
Chmury wylały wody,\pflx
zahuczały chmury\pmed
i~Twoje strzały się posypały.\pfin
Głos Twego grzmotu jak łoskot wozu,\pflx
pioruny świat rozjaśniły,\pmed
ziemia poruszyła się i~zatrzęsła.\pfin
Twoja droga wiodła przez wody,\pflx
Twoja ścieżka przez wodne obszary\pmed
i~nie znać było Twych śladów.\pfin
Wiodłeś Twój lud jak trzodę\pmed
ręką Mojżesza i~Aarona.\pfin
\stoppsalmus

\antr

\ant[n=2] Moje serce raduje się w Panu,~* który poniża i~wywyższa.

\startpsalmus[title={Pieśń (1 Sm 2, 1-10)}]
Moje serce raduje się w~Panu,\pmed
dzięki Niemu moc moja wzrasta.\pfin
Szeroko otwarłam usta przeciw moim wrogom,\pmed
bo cieszyć się mogę Twoją pomocą.\pfin
Nikt nie jest tak święty jak Ty, Panie,\pflx
poza Tobą bowiem nie ma nikogo,\pmed
prócz naszego Boga nie ma innej ostoi.\pfin
Nie powtarzajcie słów pełnych pychy,\pmed
niech mowa harda z~ust waszych nie wychodzi,\pfin
Gdyż Pan jest Bogiem wszechwiedzącym\pmed
i~On ocenia uczynki.\pfin
Łuk potężnych się łamie,\pmed
a~mocą przepasują się słabi.\pfin
Syci za chleb się najmują,\pmed
głodni zaś odpoczywają.\pfin
Niepłodna rodzi siedmioro,\pmed
a~matka wielu dzieci usycha.\pfin
Pan daje śmierć i~życie,\pmed
wtrąca do Otchłani i~z~niej wyprowadza.\pfin
Pan czyni ubogim lub bogatym,\pmed
poniża i~wywyższa.\pfin
Biedaka z~prochu podnosi,\pmed
z~błota dźwiga nędzarza,\pfin
By go wśród książąt posadzić\pmed
i~dać mu tron chwały.\pfin
Fundamenty ziemi należą do Pana\pmed
i~na nich świat On położył.\pfin
On strzeże kroków swoich wiernych,\pflx
grzesznicy zaś zginą w~ciemnościach,\pmed
bo nie własną siłą człowiek zwycięża.\pfin
Pan wniwecz opornych obraca\pmed
i~przeciw nim grzmi na niebiosach.\pfin
Pan sądzi krańce ziemi,\pflx
króla obdarza potęgą\pmed
i~wywyższa moc swego pomazańca.\pfin
\stoppsalmus

\antr

\ant[n=3] Pan króluje,~* wesel się, ziemio.~\cont

\startpsalmus[title={Psalm 97}]
Pan króluje, wesel się, ziemio,\pmed
\cont~radujcie się, liczne wyspy!\pfin
Obłok i~ciemność wokół Niego,\pmed
prawo i~sprawiedliwość podstawą Jego tronu.\pfin
Przed Jego obliczem idzie ogień\pmed
i~dokoła pożera nieprzyjaciół Jego.\pfin
Jego błyskawice wszechświat rozświetlają,\pmed
a~ziemia drży na ten widok.\pfin
Góry jak wosk topnieją przed obliczem Pana,\pmed
przed obliczem Władcy całej ziemi.\pfin
Jego sprawiedliwość rozgłaszają niebiosa\pmed
i~wszystkie ludy widzą Jego chwałę.\pfin
Niech zawstydzą się wszyscy, którzy czczą posągi\pflx
i~chlubią się bożkami.\pmed
Niech wszystkie bóstwa hołd Mu oddają!\pfin
Słyszy o~tym i~cieszy się Syjon,\pflx
radują się miasta Judy\pmed
z~Twoich wyroków, o~Panie.\pfin
Ponad całą ziemią Tyś bowiem wywyższony\pmed
i~nieskończenie wyższy od wszystkich bogów.\pfin
Pan tych miłuje, którzy zła nienawidzą,\pflx
On strzeże dusz świętych swoich,\pmed
wydziera je z~rąk grzeszników.\pfin
Światło wschodzi dla sprawiedliwego\pmed
i~radość dla ludzi prawego serca.\pfin
Weselcie się w~Panu, sprawiedliwi,\pmed
i~sławcie Jego święte imię.\pfin
\stoppsalmus

\antr

\starthourpart[title={Czytanie\hfill\tf Wj 6,2-6}]
Bóg rozmawiał z~Mojżeszem i~powiedział mu: Jam jest Jahwe. Ja objawiłem się
Abrahamowi, Izaakowi i~Jakubowi jako Bóg Wszechmocny, ale imienia mego, Jahwe,
nie objawiłem im. Ponadto ustanowiłem też przymierze moje z~nimi, że im dam kraj
Kanaan, kraj ich wędrówek, gdzie przebywali jako przybysze. Ja także usłyszałem
jęk Izraelitów, których Egipcjanie obciążyli robotami, i~wspomniałem na moje
przymierze. Przeto powiedz synom izraelskim: Ja jestem Pan! Uwolnię was od
jarzma egipskiego i~wybawię was z~niewoli, i~wyswobodzę was wyciągniętym
ramieniem i~przez surowe kary.

\starthourpart[title={Pieśń Zachariasza}]
\ant[title={Ant. do pieśni Zachariasza}]
Gdy Elżbieta usłyszała pozdrowienie Maryi,~* wydała okrzyk i powiedziała:~/
A~skądże mi to, że Matka mojego Pana przychodzi do mnie.
\startrubrica
Pieśń Zachariasza jak na \at{stronie}[benedictus].
\stoprubrica

\starthourpart[title={Prośby}]

Oddając cześć naszemu Zbawicielowi, który narodził się z~Maryi Dziewicy,
zanośmy do Niego pokorne błagania:

\aklamacja Niech Twoja Matka wstawia się za nami.

\prosba Jezu, Słońce sprawiedliwości, Twoje przyjście poprzedziła Niepokalana
Dziewica, jak pełna blasku jutrzenka,
--- spraw, abyśmy zawsze żyli w~promieniach Twojej światłości.

\prosba Dozwól nam, Panie, naśladować Twoją Matkę, która obrała najlepszą
cząstkę,
--- spraw, abyśmy za Jej przykładem szukali pokarmu dającego życie wieczne.

\prosba Zbawicielu świata, Ty mocą swojego odkupienia zachowałeś Twoją Matkę
od wszelkiej zmazy grzechu,
--- zachowaj nas od skażenia grzechem.

\prosba Nasz Odkupicielu, Ty sprawiłeś, że Maryja Dziewica stała się godnym
Ciebie mieszkaniem i~przybytkiem Ducha Świętego,
--- daj, abyśmy byli na wieki świątynią Twego Ducha.

Ojcze nasz.

\starthourpart[title={Modlitwa}]
Okaż swą potęgę, Panie, i~przybądź,~\flxs niech Twoja opieka wyzwoli nas od
i~niebezpieczeństw grożących nam wskutek naszych grzechów,~\meds a~Twoja
moc niech nas zbawi. Który żyjesz i~królujesz z~Bogiem Ojcem w~jedności
Ducha Świętego,~\meds Bóg, przez wszystkie wieki wieków.

\starthourpart[title={Zakończenie \tf (piosenka dnia)}]

% Dzień  3: Bóg z nami {{{1
\page[even]
\setupheader[state=high]
\setupfooter[state=high]
\externalfigure[d03-tabernakulum][width=\textwidth]
\startday[title={Bóg z~nami}]

\starthour[title={Wezwanie}] %{{{2
\input lib/panie-otworz

\ant Chrystus nam się narodził,~* uwielbiajmy Wcielone Słowo
\startrubrica
Psalm Wezwania nr FIXME \at{str.}[invit-ps095].
\stoprubrica

\input lib/zapalenie-A

\starthour[title={Jutrznia}] %{{{2

\starthourpart[title={Hymn}]
% http://brewiarz.pl/indeksy/pokaz.php3?id=3&nr=139
% Godzina Czytań: Okres Narodzenia Pańskiego (do Objawienia Pańskiego)
% 01.01. - uroczystość Świętej Bożej Rodzicielki Maryi
% LG tom I, wydanie I (Pallottinum 1982), s. 424
% LG tom I, wydanie II (Pallottinum 2006), s. 427
% Skrócona LG (Pallottinum 1991), s. 176
\starthymnus
\item Zakwitnął korzeń Jessego,
Różdżka wydała swój Owoc,
Gdy Matka Syna zrodziła
Nadal zostając Dziewicą.
%
\item Przedwieczny Stwórca światłości
W~żłobie pozwolił się złożyć,
I~Ten, co niebo rozpostarł,
Leży spowity w~pieluszki.
%
\item Obdarzył świat swoim prawem,
Prawem dziesięciu przykazań,
A~dziś, człowiekiem się stając,
Sam jego mocy się poddał.
%
\item Już wschodzi zorza zbawienia,
Noc pierzcha, śmierć zwyciężona;
Niech cała ziemia wysławia
Bogarodzicę Maryję!
%
\item O~Chryste, Synu Dziewicy,
Tobie i~Ojcu Twojemu
Z~ożywczym Duchem jedności
Chwała niech będzie na wieki. Amen.\stophymnus

\starthourpart[title={Psalmodia}]

\ant[n=1] Rano głosimy Twą łaskawość, Panie, * a wierność Twoją nocami.

\startpsalmus[title={Psalm 92}]
Dobrze jest dziękować Panu,\pmed
śpiewać Twojemu imieniu, Najwyższy,\pfin
Rano głosić łaskawość Twoją,\pmed
a~wierność Twoją nocami,\pfin
Na harfie dziesięciostrunnej i~lirze,\pmed
pieśnią przy dźwiękach cytry.\pfin
Bo weselę się, Panie, Twoimi czynami,\pmed
raduję się dziełami rąk Twoich.\pfin
Jak wielkie są dzieła Twoje, Panie,\pmed
i~jakże głębokie Twe myśli!\pfin
Nie zna ich człowiek nierozumny\pmed
i~głupiec ich nie pojmuje.\pfin
Chociaż występni się pienią jak zielsko,\pmed
a~złoczyńcy jaśnieją przepychem,\pfin
I~tak pójdą na wieczną zagładę,\pmed
Ty zaś, Panie, na wieki jesteś wywyższony.\pfin
Bo oto wrogowie Twoi, Panie,\pflx
bo oto wrogowie Twoi poginą,\pmed
rozproszą się wszyscy złoczyńcy.\pfin
Dałeś mi siłę bawołu,\pmed
skropiłeś mnie świeżym olejkiem.\pfin
Moje oko spogląda z~góry na nieprzyjaciół,\pflx
a~uszy me usłyszały o~klęsce przeciwników,\pmed
tych, którzy na mnie powstają.\pfin
Sprawiedliwy zakwitnie jak palma,\pmed
rozrośnie się jak cedr na Libanie.\pfin
Zasadzeni w~domu Pańskim\pmed
rozkwitną na dziedzińcach Boga naszego.\pfin
Nawet i~w~starości wydadzą owoc,\pmed
zawsze pełni życiodajnych soków,\pfin
Aby świadczyć, że Pan jest sprawiedliwy;\pmed
On moją Opoką i~nie ma w~Nim nieprawości.\pfin
\stoppsalmus

\antr

\ant[n=2][join] Uznajcie wielkość~* naszego Boga.

\startpsalmus[title={Pieśń (Pwt 32, 1-12)}]
Uważajcie, niebiosa, na to, co powiem,\pmed
słuchaj, ziemio, głosu mojego.\pfin
Jak deszcz niech spływa moje pouczenie,\pmed
jak rosa niech pada me słowo,\pfin
Jak deszcz rzęsisty na zieleń,\pmed
jak deszcz życiodajny na trawę!\pfin
Oto będę głosić imię Pana;\pmed
uznajcie wielkość naszego Boga!\pfin
On jest Opoką, a~Jego dzieło doskonałe,\pmed
wszystkie Jego drogi są słuszne.\pfin
On jest Bogiem wiernym i~nie zawodzi,\pmed
On sprawiedliwy i~prawy.\pfin
Przestali być Jego dziećmi,\pflx
bo grzech popełnili,\pmed
pokolenie przewrotne i~zakłamane.\pfin
Więc tak chcesz odpłacić Panu,\pmed
ludu głupi i~bezrozumny?\pfin
Czyż nie On jest twoim Ojcem i~Stwórcą,\pmed
który cię uczynił i~dał ci życie?\pfin
Wspomnij na dni, które przeminęły,\pmed
rozważ lata poprzednich pokoleń.\pfin
Zapytaj swego ojca, by cię pouczył,\pmed
i~twoich starców, niech ci opowiedzą,\pfin
Jak to Najwyższy obdarzał dziedzictwem narody\pmed
i~rozdzielał synów człowieczych.\pfin
Wtedy wytyczył granice dla ludów\pmed
według liczby synów Boga sprawiedliwego,\pfin
Bo Jego lud jest własnością Pana,\pmed
Jakub Jego wyłącznym dziedzictwem.\pfin
Znalazł go na ziemi pustynnej,\pmed
na odludziu, gdzie brzmiały tylko dzikie głosy.\pfin
Opiekował się nim i~pouczał,\pmed
strzegł jak źrenicy oka.\pfin
Jak orzeł, który krąży nad gniazdem,\pmed
by z~niego wywabić swe pisklęta,\pfin
I~bierze je na skrzydła rozpostarte,\pmed
niosąc je na samym sobie,\pfin
Tak Pan go prowadził,\pmed
a~nie było z~nim obcego boga.\pfin
\stoppsalmus

\antr

\ant[n=3] Panie, jak przedziwne jest Twoje imię~* na całej ziemi.

\startpsalmus[title={Psalm 8}]
O~Panie, nasz Panie,\pflx
jak przedziwne jest Twoje imię na całej ziemi!\pmed
Tyś swój majestat wyniósł nad niebiosa.\pfin
Sprawiłeś, że na przekór Twoim przeciwnikom\pflx
usta dzieci i~niemowląt oddają Ci chwałę,\pmed
aby poskromić nieprzyjaciela i~wroga.\pfin
Gdy patrzę na Twe niebo, dzieło palców Twoich,\pmed
na księżyc i~gwiazdy, któreś Ty utwierdził:\pfin
Czym jest człowiek, że o~nim pamiętasz,\pmed
czym syn człowieczy, że troszczysz się o~niego?\pfin
Uczyniłeś go niewiele mniejszym od aniołów,\pmed
uwieńczyłeś go czcią i~chwałą.\pfin
Obdarzyłeś go władzą nad dziełami rąk Twoich,\pmed
wszystko złożyłeś pod jego stopy:\pfin
Owce i~bydło wszelakie,\pmed
i~dzikie zwierzęta,\pfin
Ptaki niebieskie i~ryby morskie,\pmed
wszystko, co szlaki mórz przemierza.\pfin
O~Panie, nasz Panie,\pmed
jak przedziwne jest Twoje imię na całej ziemi!\pfin
\stoppsalmus

\antr

\starthourpart[title={Czytanie\hfill\tf Mt 1,20-23}]
Gdy Józef powziął tę myśl, oto anioł Pański ukazał mu się we śnie i~rzekł:
Józefie, synu Dawida, nie bój się wziąć do siebie Maryi, twej Małżonki; albowiem
z~Ducha Świętego jest to, co się w~Niej poczęło. Porodzi Syna, któremu nadasz
imię Jezus, On bowiem zbawi swój lud od jego grzechów. A~stało się to wszystko,
aby się wypełniło słowo Pańskie powiedziane przez Proroka: Oto Dziewica pocznie
i~porodzi Syna, któremu nadadzą imię Emmanuel, to znaczy: Bóg z~nami.

\starthourpart[title={Pieśń Zachariasza}]
\ant[title={Ant do pieśni Zachariasza}] Chwała na wysokościach Bogu,~* a~na
ziemi pokój ludziom, których umiłował,~/ alleluja.
\startrubrica
Pieśń Zachariasza jak na \at{stronie}[benedictus].
\stoprubrica


\starthourpart[title={Prośby}]

Uwielbiajmy Słowo Boże, które istniejąc od wieków, w~pełni czasów stało
się ciałem i~zamieszkało między nami. Z~radością więc wołajmy do Niego:

\aklamacja Głośmy z~weselem: Bóg jest między nami.

\prosba Chryste, Słowo Przedwieczne, Ty przychodząc na ziemię opromieniasz ją
radością,
--- napełnij nasze serca łaską Twego nawiedzenia.

\prosba Nasz Zbawicielu, Ty przez swe narodzenie objawiasz nam wierność Boga,
--- spraw, abyśmy wiernie dochowali przyrzeczeń chrztu świętego.

\prosba Królu nieba i~ziemi, Ty posłałeś aniołów, aby ludziom głosili pokój,
--- zachowaj nasze życie w~Twoim pokoju.

\prosba Panie, Ty przyszedłeś jako prawdziwy krzew winny dający nam życie,
--- spraw, abyśmy jak gałązki trwali zawsze w~Tobie i~przynosili obfite owoce.

Ojcze nasz.

\starthourpart[title={Modlitwa}]

Boże, Ty w~przedziwny sposób stworzyłeś człowieka i~w~jeszcze cudowniejszy
sposób odnowiłeś jego godność,~\flxs daj nam uczestniczyć w~bóstwie Twojego
Syna,~\meds który przyjął naszą ludzką naturę. Który z~Tobą żyje i~króluje
w~jedności Ducha Świętego,~\meds Bóg, przez wszystkie wieki wieków.

\starthourpart[title={Zakończenie \tf (piosenka dnia)}]

% Dzień  4: Bóg wzywa {{{1
\page[even]
\setupheader[state=high]
\setupfooter[state=high]
\externalfigure[d04-amen][width=\textwidth]
\startday[title={Bóg wzywa}]

\starthour[title={Wezwanie}] %{{{2
\input lib/panie-otworz

\ant Chrystus nam się objawił,~* pokłon Jemu oddajmy
\startrubrica
Psalm Wezwania nr FIXME \at{str.}[invit-ps095].
\stoprubrica

\input lib/zapalenie-A

\starthour[title={Jutrznia}] %{{{2

\starthourpart[title={Hymn}]
% Jutrznia: 2 lutego (Ofiarowanie Pańskie)
% Skrócona LG (Pallottinum 1991), s. 1209
\starthymnus
\item Przystrój Syjonie, świątynię,
Jeżeli czekasz na Pana,
Jasnością wiary natchniony
Otwórz tym dwojgu podwoje.
%
\item Przybądź, o starcze szczęśliwy,
Niech ziści się obietnica,
I głoś narodom z weselem:
Oto jest Światło dla pogan!
%
\item Wnoszą rodzice Chrystusa
I Ten, co sam jest świątynia,
W świątyni poddał się Prawu,
Chociaż sam mu nie podlega.
%
\item Panno ofiaruj swe Dziecię:
To Twój i Ojca Jedyny,
I On w ofierze nas złoży.
Stając się ceną zbawienia.
%
\item Pójdź, o królewska Dziewico,
I Syna z darem ofiaruj
Radością wszystkich napełnia
Ten, który wszystkich odkupi.
%
\item Chwała i cześć Tobie, Jezu,
Narodom dziś objawiony,
I Ojcu z Duchem Najświętszym
Teraz i zawsze niech będzie. Amen.\stophymnus

\starthourpart[title={Psalmodia}]

\ant[n=1] Gotowe jest serce moje, Boże,~* gotowe jest serce moje,~/ zaśpiewam
psalm i~zagram.~\cont

\startpsalmus[title={Psalm 108}]
Gotowe jest serce moje, Boże,\pflx
gotowe jest serce moje,\pmed
zaśpiewam psalm i~zagram,\pfin
\cont~Zbudź się, duszo moja,\pflx
zbudź się, harfo i~cytro,\pmed
a~ja obudzę jutrzenkę.\pfin
Będę Cię chwalił wśród ludów, Panie,\pmed
zaśpiewam Ci psalm wśród narodów.\pfin
Bo Twoja łaska sięga aż do nieba,\pmed
a~wierność Twoja po chmury.\pfin
Wznieś się ponad niebiosa, Panie,\pmed
nad całą ziemię Twoja chwała.\pfin
Aby ocaleli, których Ty miłujesz,\pmed
wspomóż nas Twoją prawicą i~wysłuchaj.\pfin
Bóg przemówił w~swojej świątyni:\pflx
„Będę się radował i~podzielę Sychem,\pmed
a~dolinę Sukkot wymierzę.\pfin
Do mnie należy Gilead i~ziemia Manassesa,\pflx
Efraim jest szyszakiem mej głowy,\pmed
Juda berłem moim.\pfin
Moab jest moją misą do mycia,\pflx
na Edomie mój but postawię,\pmed
zatriumfuję nad Filisteą!”\pfin
Któż mnie wprowadzi do miasta warownego?\pmed
Któż mnie doprowadzi do Edomu?\pfin
Czyż nie Ty, Boże, który nas odrzuciłeś\pmed
i~już nie wychodzisz, Boże, z~naszymi wojskami?\pfin
Daj nam pomoc przeciw nieprzyjacielowi,\pmed
bo ludzkie wsparcie jest zawodne.\pfin
Dokonamy w~Bogu czynów pełnych mocy,\pmed
a~On podepcze naszych nieprzyjaciół.\pfin
\stoppsalmus

\antr

\ant[n=2] Dam wam serce nowe~* i~nowego ducha tchnę w~wasze wnętrze.

\startpsalmus[title={Pieśń (Ez 36, 24-28)}]
Wezmę was spośród ludów,\pflx
zgromadzę was ze wszystkich krajów\pmed
i~przywiodę z~powrotem do waszej ziemi.\pfin
Pokropię was czystą wodą,\pmed
i~staniecie się czyści.\pfin
Obmyję was z~wszelkiej nieczystości\pmed
i~z~waszego bałwochwalstwa.\pfin
Dam wam serce nowe\pmed
i~nowego ducha tchnę w~wasze wnętrze.\pfin
Wyjmę z~was serce kamienne,\pmed
i~dam wam serce z~ciała.\pfin
Tchnę w~was mojego Ducha\pmed
i~sprawię, że będziecie żyć według mych nakazów,\pfin
Że będziecie przestrzegać przykazań\pmed
i~postępować zgodnie z~nimi.\pfin
Wtedy zamieszkacie w~kraju,\pmed
który dałem waszym przodkom,\pfin
I~będziecie moim ludem,\pmed
Ja zaś będę Bogiem waszym.\pfin
\stoppsalmus

\antr

\ant[n=3][join] Będę chwalił Pana~* do końca mego życia.

\startpsalmus[title={Psalm 146}]
Chwal, duszo moja, Pana;\pflx
będę chwalił Pana do końca mego życia,\pmed
będę śpiewał mojemu Bogu, dopóki istnieję.\pfin
Nie pokładajcie ufności w~książętach\pmed
ani w~człowieku, który zbawić nie może.\pfin
Kiedy duch go opuści, znów w~proch się obraca\pmed
i~przepadają wszystkie jego zamiary.\pfin
Szczęśliwy ten, kogo wspiera Bóg Jakuba,\pmed
kto pokłada nadzieję w~Panu Bogu.\pfin
On stworzył niebo i~ziemię, i~morze\pmed
ze wszystkim, co w~nich istnieje.\pfin
On wiary dochowuje na wieki,\pmed
uciśnionym sprawiedliwość wymierza,\pfin
Chlebem karmi głodnych,\pmed
wypuszcza na wolność uwięzionych.\pfin
Pan przywraca wzrok ociemniałym,\pflx
Pan dźwiga poniżonych,\pmed
Pan kocha sprawiedliwych.\pfin
Pan strzeże przybyszów,\pflx
ochrania sierotę i~wdowę,\pmed
lecz występnych kieruje na bezdroża.\pfin
Pan króluje na wieki,\pmed
Bóg twój, Syjonie, przez pokolenia.\pfin
\stoppsalmus

\antr

\starthourpart[title={Czytanie\hfill\tf Wj 3,4}]
Gdy zaś Pan ujrzał, że Mojżesz podchodził, żeby się przyjrzeć, zawołał Bóg
do niego ze środka krzewu: Mojżeszu, Mojżeszu! On zaś odpowiedział: Oto
jestem.

\starthourpart[title={Pieśń Zachariasza}]
\ant[title={Ant. do pieśni Zachariasza}]
Gdy rodzice wnosili do świątyni Dzieciątko Jezus,~* Symeon wziął Je w objęcia
i~błogosławił Boga.
\startrubrica
Pieśń Zachariasza jak na \at{stronie}[benedictus].
\stoprubrica

\starthourpart[title={Prośby}]

Uwielbiajmy naszego Zbawiciela, który został ofiarowany Bogu w~świątyni,
i~zanośmy do Niego nasze prośby:

\aklamacja Niech nasze oczy ujrzą Twe zbawienie.

\prosba Jezu Chryste, Ty zechciałeś, aby zgodnie z~Prawem ofiarowano Cię
w~świątyni,
--- naucz nas samych siebie składać w~ofierze Kościoła razem z~Tobą.

\prosba Jezu, Pociecho Izraela, sprawiedliwy Symeon przyszedł do świątyni na
Twoje spotkanie,
--- spraw, abyśmy spotykali Ciebie w~naszych braciach.

\prosba Jezu, Nadziejo narodów, o~Tobie prorokini Anna mówiła wszystkim, którzy
oczekiwali odkupienia Izraela,
--- naucz nas godnie o~Tobie mówić do wszystkich ludzi.

\prosba Jezu, kamieniu węgielny królestwa Bożego, Ty zostałeś postawiony jako
znak sprzeciwu,
--- spraw, aby wszyscy ludzie przez wiarę i~miłość osiągnęli z~Tobą chwałę
zmartwychwstania.

Ojcze nasz.

\starthourpart[title={Modlitwa}]
Boże, Ty kierujesz swoim ludem przez pasterzy,~\flxs ześlij na swój Kościół
ducha pobożności i~męstwa~\meds i~powołaj ludzi, którzy będą godnie pełnili
służbę przy ołtarzu oraz mężnie i~pokornie głosili Ewangelię. Przez naszego Pana
Jezusa Chrystusa, Twojego Syna,~\flxs który z~Tobą żyje i~króluje w~jedności
Ducha Świętego,~\meds Bóg, przez wszystkie wieki wieków.

\starthourpart[title={Zakończenie \tf (piosenka dnia)}]

% Dzień  5: Bóg posyła {{{1
\page[even]
\setupheader[state=high]
\setupfooter[state=high]
\externalfigure[d05-wlozenie-rak][width=\textwidth]
\startday[title={Bóg posyła}]

\starthour[title={Wezwanie}] %{{{2
\input lib/panie-otworz

% TODO tu gdzieś kreska?
\ant Uwielbiajmy Chrystusa, umiłowanego Syna Bożego, w~którym Ojciec upodobał
sobie
\startrubrica
Psalm Wezwania nr FIXME \at{str.}[invit-ps095].
\stoprubrica

\input lib/zapalenie-A

\starthour[title={Jutrznia}] %{{{2

\starthourpart[title={Hymn}]
% http://brewiarz.pl/indeksy/pokaz.php3?id=3&nr=038
% Gdy Jan wypełniał swe zadanie
% Godzina Czytań: Okres Narodzenia Pańskiego (po Objawieniu Pańskim)
% święto Chrztu Pańskiego (niedziela po 6 stycznia)
% II Nieszpory: Okres Narodzenia Pańskiego (po Objawieniu Pańskim)
% święto Chrztu Pańskiego (niedziela po 6 stycznia)
% LG tom I, wydanie I (Pallottinum 1982), s. 551, 561
% LG tom I, wydanie II (Pallottinum 2006), s. 551, 561
% Skrócona LG (Pallottinum 1991), s. 242
\starthymnus
\item Gdy Jan wypełniał swe zadanie,
Wszedł do Jordanu Stwórca świata,
By swym obmyciem w~jego wodach
Czystość przywrócić nurtom rzeki.
%
\item Nie potrzebował chrztu pokuty
Ten, który zrodził się z~Dziewicy,
Lecz przyjął go, by nasze grzechy
Zgładzić przez własne zanurzenie.
%
\item Obwieścił Ojciec głosem z~nieba:
„Oto mój Syn umiłowany”,
A~Duch w~postaci gołębicy
Przyszedł z~wysoka na Jezusa.
%
\item Jaśnieje dzisiaj dla Kościoła
Jego zbawienna tajemnica,
Bo w~Trzech Osobach Bóg się jawi,
Który jednością jest na wieki.
%
\item Niech Tobie, Chryste, brzmi podzięka
Za to, że ludziom objawiłeś
Majestat swój i~chwałę równą
Z~Ojcem i~Duchem Przenajświętszym. Amen.\stophymnus

\starthourpart[title={Psalmodia}]

\ant[n=1] Tobie chcę śpiewać, o~Panie,~* będę szedł drogą nieskalaną.

\startpsalmus[title={Psalm 101}]
Będę śpiewał o~sprawiedliwości i~łasce,\pmed
Tobie chcę śpiewać, o~Panie.\pfin
Będę szedł drogą nieskalaną;\pmed
kiedyż Ty do mnie przyjdziesz?\pfin
Będę chodził z~sercem niewinnym\pmed
wewnątrz swojego domu.\pfin
Nie będę zwracał oczu\pmed
ku sprawie niegodziwej.\pfin
W~nienawiści mam czyny przestępcy,\pmed
nie przylgną one do mnie.\pfin
Przewrotne serce będzie ode mnie z~daleka,\pmed
tego, co złe, nawet znać nie chcę.\pfin
Zniszczę każdego, kto skrycie uwłacza bliźniemu,\pmed
pysznych oczu i~serca nadętego nie zniosę.\pfin
Oczy swe zwracam na wiernych tej ziemi,\pmed
aby ze mną zamieszkać mogli.\pfin
Ten, który kroczy drogą nieskalaną,\pmed
będzie mi usługiwał.\pfin
Kto knuje podstęp, nie zamieszka w~mym domu,\pmed
nie ostoi się wobec mego wzroku, kto rozgłasza kłamstwa.\pfin
Każdego dnia będę tępił wszystkich grzeszników ziemi,\pmed
wypędzę z~miasta Pańskiego wszelkich złoczyńców.\pfin
\stoppsalmus

\antr

\ant[n=2][join] Nie odbieraj nam, Panie,~* swego miłosierdzia.

\startpsalmus[title={Pieśń (Dn 3,26-27. 29. 34-41)}]
Błogosławiony jesteś, Panie, Boże naszych ojców,\pflx
i~pełen chwały,\pmed
a~Twoje imię jest błogosławione na wieki.\pfin
Jesteś sprawiedliwy we wszystkim, coś uczynił z~nami,\pmed
wszelkie Twoje dzieła są pełne prawdy.\pfin
Twoje drogi są proste,\pmed
a~wyroki najsłuszniejsze.\pfin
Myśmy zgrzeszyli i~popełnili nieprawość\pmed
odstępując od Ciebie.\pfin
We wszystkim przewrotność okazaliśmy\pmed
i~nie słuchaliśmy Twoich przykazań.\pfin
Nie opuszczaj nas na zawsze\pflx
i~ze względu na świętość Twego imienia\pmed
nie zrywaj Twojego przymierza.\pfin
Nie odbieraj nam swego miłosierdzia\pmed
przez wzgląd na Abrahama, przyjaciela Twego,\pfin
Twojego sługę, Izaaka,\pmed
i~Twego świętego, Izraela.\pfin
Im to przyrzekłeś,\pmed
że rozmnożysz ich potomstwo\pfin
Jak gwiazdy na niebie\pmed
i~jak piasek na morskim brzegu.\pfin
Panie, oto jesteśmy najmniejsi\pmed
spośród wszystkich narodów.\pfin
Oto dziś jesteśmy poniżeni na całej ziemi\pmed
z~powodu naszych grzechów.\pfin
Nie ma już władcy, proroka ani wodza,\pflx
ani całopalenia, ani ofiar,\pmed
ani daru pokarmów, ani kadzenia.\pfin
Nie ma gdzie złożyć Tobie daru z~pierwszych płodów\pmed
i~doświadczyć miłosierdzia Twego.\pfin
Niech jednak dusza strapiona i~duch uniżony\pmed
znajdą upodobanie u~Ciebie.\pfin
Tak jak całopalenia z~baranów i~cielców,\pmed
i~tysięcy tłustych owiec,\pfin
Niech się dziś stanie nasza ofiara dla Ciebie\pmed
i~spodoba się Tobie.\pfin
Bo ci, którzy ufność pokładają w~Tobie,\pmed
nie zaznają wstydu.\pfin
Teraz zaś z~całego serca idziemy za Tobą,\pflx
odczuwamy lęk wobec Ciebie\pmed
i~szukamy Twojego oblicza.\pfin
\stoppsalmus

\antr

\ant[n=3] Boże, będę Ci śpiewał~* pieśń nową.

\startpsalmus[title={Psalm 144,1-10}]
Błogosławiony Pan, Opoka moja,\pflx
On moje ręce zaprawia do walki,\pmed
moje palce do bitwy.\pfin
On mocą i~warownią moją,\pmed
osłoną moją i~moim wybawcą,\pfin
Moją tarczą i~schronieniem,\pmed
On, który mi poddaje ludy.\pfin
Panie, czym jest człowiek, że troszczysz się o~niego,\pmed
czym syn człowieczy, że Ty o~nim myślisz?\pfin
Do tchnienia wiatru podobny jest człowiek,\pmed
dni jego jak cień przemijają.\pfin
Nachyl swych niebios i~zstąp po nich, Panie,\pmed
gór dotknij, aby zadymiły.\pfin
Rozprosz ich uderzeniem pioruna,\pmed
wypuść swe strzały, by ich porazić.\pfin
Wyciągnij swą rękę z~wysoka,\pmed
wybaw mnie z~wód ogromnych.\pfin
Uwolnij z~rąk cudzoziemców,\pflx
tych, których usta mówią kłamliwie,\pmed
a~prawica wznosi się do fałszywej przysięgi.\pfin
Boże, będę Ci śpiewał pieśń nową,\pmed
grać Ci będę na harfie o~dziesięciu strunach.\pfin
Ty królom dajesz zwycięstwo,\pmed
Tyś wyzwolił Dawida, swego sługę.\pfin
\stoppsalmus

\antr

\starthourpart[title={Czytanie\hfill\tf Iz 6,8-9a}]
I~usłyszałem głos Pana mówiącego: Kogo mam posłać? Kto by Nam poszedł?
Odpowiedziałem: Oto ja, poślij mnie! I~rzekł mi: Idź i~mów do tego ludu.

\starthourpart[title={Pieśń Zachariasza}]
\ant[title={Ant. do pieśni Zachariasza:}]
Chrystus chrztem swoim uświęcił świat cały,~* udzielił nam odpuszczenia
grzechów~/ i~przez wodę i~Ducha oczyścił nas wszystkich.
\startrubrica
Pieśń Zachariasza jak na \at{stronie}[benedictus].
\stoprubrica

\starthourpart[title={Prośby}]

Błagajmy naszego Odkupiciela, który zechciał przyjąć chrzest z~rąk Jana
w~Jordanie, i~wołajmy do Niego:

\aklamacja Panie, zmiłuj się nad nami!

\prosba Chryste, Ty przez swe objawienie opromieniłeś nas swoim światłem,
--- spraw, abyśmy je dziś nieśli spotykanym ludziom.

\prosba Chryste, Ty przyjmując chrzest od swojego sługi, uniżyłeś się, aby dać
nam przykład pokory,
--- spraw, byśmy umieli pokornie służyć braciom.

\prosba Chryste, Ty przez swój chrzest obmyłeś nas z~wszelkiego grzechu
i~przywróciłeś nam godność dzieci Twojego Ojca,
--- udziel ducha przybrania wszystkim, którzy Cię szukają.

\prosba Chryste, Ty przez swój chrzest uświęciłeś całe stworzenie i~otworzyłeś
ochrzczonym bramę nawrócenia,
--- spraw, abyśmy byli w~świecie sługami Twojej Ewangelii.

\prosba Chryste, Ty przez swój chrzest objawiłeś nam Trójcę Świętą,
--- odnów ducha przybranych dzieci w~kapłańskim ludzie ochrzczonych.

Ojcze nasz.

\starthourpart[title={Modlitwa}]
Boże, Ty chcesz, aby wszyscy ludzie zostali zbawieni i~doszli do poznania
prawdy,~\flxs wejrzyj na swoje wielkie żniwo i~poślij na nie robotników, aby
głosili Ewangelię wszelkiemu stworzeniu,~\meds niech Twój lud zgromadzony przez
Słowo Życia i~pokrzepiony mocą sakramentów kroczy drogą zbawienia i~miłości.
Przez naszego Pana Jezusa Chrystusa, Twojego Syna który z~Tobą żyje i~króluje
w~jedności Ducha Świętego,~\meds Bóg, przez wszystkie wieki wieków.

\starthourpart[title={Zakończenie \tf (piosenka dnia)}]

% Dzień  6: Bóg walczy {{{1
\page[even]
\setupheader[state=high]
\setupfooter[state=high]
\externalfigure[d06-popiol][width=\textwidth]
\startday[title={Bóg walczy}]

\starthour[title={Wezwanie}] %{{{2
\input lib/panie-otworz

% TODO kreska?
\ant Uwielbiajmy Chrystusa Pana, który dla nas był kuszony i~ukrzyżowany
\startrubrica
Psalm Wezwania nr FIXME \at{str.}[invit-ps095].
\stoprubrica

\input lib/zapalenie-B

\starthour[title={Jutrznia}] %{{{2

\starthourpart[title={Hymn}]
% http://brewiarz.pl/indeksy/pokaz.php3?id=3&nr=153
% Nieszpory: Wielki Post
% w oficjum dni powszednich
% LG tom II (Pallottinum 1984), s. 36
% Skrócona LG (Pallottinum 1991), s. 246
\starthymnus
\item Dałeś nam przykład, o~Jezu,
Czterdziestu dni umartwienia;
Aby w~nas ducha odnowić,
Wymagasz postu i~skruchy.
%
\item Stań więc pośrodku Kościoła
I~spójrz na jego pokutę;
Ciebie pokornie prosimy,
Byś przez nią z~grzechu nas obmył.
%
\item Zło popełnione w~przeszłości
Niech Twoja łaska zniweczy;
Chroń nas od dalszych upadków
I~otocz swoją opieką.
%
\item Daj przez doroczną pokutę
Dostąpić win odpuszczenia,
Byśmy z~radością czekali
Na światło Nocy Paschalnej.
%
\item W~Trójcy Jedyny nasz Boże,
Niech Ciebie wszechświat wysławia;
Pieśń nową Tobie śpiewamy
Za odrodzenie przez łaskę. Amen.\stophymnus

\starthourpart[title={Psalmodia}]

\ant[n=1] Budzą się moje oczy jeszcze przed świtem,~* aby rozważać Twoje słowo,
Panie.

\startpsalmus[title={Psalm 119,145-152}]
Z~całego serca wołam, wysłuchaj mnie, Panie,\pmed
zachować chcę Twoje ustawy\pfin
Wołam do Ciebie, a~Ty mnie wybaw,\pmed
będę strzegł Twoich napomnień.\pfin
Przychodzę o~świcie i~wołam,\pmed
pokładam ufność w~Twoich słowach.\pfin
Budzą się moje oczy jeszcze przed świtem,\pmed
aby rozważać Twoje słowo.\pfin
W~dobroci swej, Panie, słuchaj głosu mego\pmed
i~daj mi życie zgodne z~Twoim przykazaniem.\pfin
Zbliżają się niegodziwi moi prześladowcy,\pmed
dalecy są oni od Twojego Prawa.\pfin
Jesteś blisko, Panie,\pmed
i~wszystkie Twe przykazania są prawdą.\pfin
Od dawna wiem z~Twoich napomnień,\pmed
że ustaliłeś je na wieki.\pfin
\stoppsalmus

\antr

\ant[n=2] Pan jest moją mocą i~źródłem męstwa,~* Jemu zawdzięczam moje ocalenie.

\startpsalmus[title={Pieśń (Wj 15, 1b-13.17-18)}]
Zaśpiewam na cześć Pana, który okrył się sławą,\pmed
gdy konia i~jeźdźca pogrążył w~morskiej toni.\pfin
Pan jest moją mocą i~źródłem męstwa,\pmed
Jemu zawdzięczam moje ocalenie.\pfin
On Bogiem moim, uwielbiać Go będę,\pmed
On Bogiem mego ojca, będę Go wywyższać.\pfin
Pan, wojownik potężny,\pmed
«Ten, który jest», brzmi Jego imię.\pfin
Rzucił w~morze rydwany faraona i~wojska jego,\pmed
wybrani wodzowie legli w~Morzu Czerwonym.\pfin
Ogarnęły ich przepaści,\pmed
jak głaz runęli w~głębinę.\pfin
Okazała się wtedy potęga Twej prawicy, Panie,\pmed
prawica Twoja, Panie, zmiażdżyła nieprzyjaciół.\pfin
Pełen mocy zniszczyłeś swoich wrogów,\pmed
wybuchnąłeś gniewem, co spalił ich jak słomę.\pfin
Od Twojego tchnienia spiętrzyły się wody,\pflx
jak wał stanęły ogromne fale,\pmed
w~pośrodku morza zakrzepły otchłanie.\pfin
Mówił nieprzyjaciel: „Będę ścigał, dopadnę,\pflx
rozdzielę łupy i~nasycę mą duszę,\pmed
dobędę miecza, moja ręka ich zniszczy!”\pfin
Lecz Tyś posłał swój wicher i~przykryło ich morze,\pmed
zatonęli jak ołów wśród gwałtownych wirów.\pfin
Któż spomiędzy bogów równy Tobie, Panie,\pflx
kto blaskiem świętości podobny do Ciebie,\pmed
straszliwy w~czynach, sprawiający cuda?\pfin
Wyciągnąłeś prawicę\pmed
i~wchłonęła ich ziemia.\pfin
Ty zaś wiodłeś swą łaską lud oswobodzony\pflx
i~doprowadziłeś go Twoją mocą*\pfin
do swego mieszkania świętego.\pfin
Wprowadziłeś ich i~osadziłeś na górze Twojego dziedzictwa,\pmed
w~miejscu, któreś uczynił swoim mieszkaniem,\pfin
W~świątyni zbudowanej Twoimi rękami;\pmed
Pan jest Królem na zawsze i~na wieki.\pfin
\stoppsalmus

\antr

\ant[n=3][join] Chwalcie Pana,~* wszystkie narody.~\cont

\startpsalmus[title={Psalm 117}]
Chwalcie Pana, wszystkie narody,\pmed
\cont~wysławiajcie Go, wszystkie ludy!\pfin
Bo potężna nad nami Jego łaska,\pmed
a~wierność Pana trwa na wieki.\pfin
\stoppsalmus

\antr

\starthourpart[title={Czytanie\hfill\tf Wj 14,13-14}]
Mojżesz odpowiedział ludowi: Nie bójcie się! Pozostańcie na swoim miejscu,
a~zobaczycie zbawienie od Pana, jakie zgotuje nam dzisiaj. Egipcjan, których
widzicie teraz, nie będziecie już nigdy oglądać. Pan będzie walczył za was, a~wy
będziecie spokojni.

\starthourpart[title={Pieśń Zachariasza}]
\ant[title={Ant. do pieśni Zachariasza:}]
Duch wyprowadził Jezusa na pustynię,~* aby był kuszony przez diabła.~/ A~gdy
przepościł czterdzieści dni i~nocy,~/ głód potem odczuwał.
\startrubrica
Pieśń Zachariasza jak na \at{stronie}[benedictus].
\stoprubrica

\starthourpart[title={Prośby}]

Błogosławmy naszego Odkupiciela, który łaskawie wysłużył nam ten czas
zbawienia. Z~pokorą zanośmy do Niego nasze prośby:

\aklamacja Stwórz w~nas, o~Panie, nowego ducha.

\prosba Chryste, nasze życie, Ty zechciałeś, abyśmy przez chrzest zostali
pogrzebani z~Tobą w~śmierci i~z~Tobą zostali wskrzeszeni,
--- dopomóż nam dzisiaj postępować w~nowości życia.

\prosba Panie, Ty wszystkim dobrze czyniłeś,
--- spraw, abyśmy troszczyli się o~wspólne dobro wszystkich ludzi.

\prosba Daj, abyśmy zgodnie współpracowali w~doczesnej społeczności,
--- a~zarazem zdążali do wiecznej ojczyzny.

\prosba Chryste, lekarzu ciał i~dusz, ulecz rany naszego serca,
--- abyśmy stale korzystali z~Twej pomocy na drodze świętości.

Ojcze nasz.

\starthourpart[title={Modlitwa}]
Wszechmogący, wieczny Boże, wejrzyj na naszą słabość w~walce z~mocami
ciemności~\meds i~wyciągnij w~naszej obronie swoją potężną prawicę. Przez
naszego Pana Jezusa Chrystusa, Twojego Syna,~\flxs który z~Tobą żyje
i~króluje w~jedności Ducha Świętego,~\meds Bóg, przez wszystkie wieki
wieków.

\starthourpart[title={Zakończenie \tf (piosenka dnia)}]

% Dzień  7: Plagi {{{1
\page[even]
\setupheader[state=high]
\setupfooter[state=high]
\externalfigure[d07-bicie-w-piersi][width=\textwidth]
\startday[title={Plagi}]

\starthour[title={Wezwanie}] %{{{2
\input lib/panie-otworz

% TODO kreska?
\ant Jeśli głos Pana usłyszycie, nie zatwardzajcie serc waszych
\startrubrica
Psalm Wezwania nr FIXME \at{str.}[invit-ps095].
\stoprubrica

\input lib/zapalenie-B

\starthour[title={Jutrznia}] %{{{2

\starthourpart[title={Hymn}]
% http://brewiarz.pl/indeksy/pokaz.php3?id=3&nr=205
% Jutrznia: Wielki Post
% w oficjum niedzielnym
% LG tom II (Pallottinum 1984), s. 39
% Skrócona LG (Pallottinum 1991), s. 249
\starthymnus
\item Padnijmy twarzą na ziemię,
Błagajmy Boga ze skruchą,
Przed Sędzią łzy wylewajmy,
By Jego gniew się uśmierzył.
%
\item Tak często, Ojcze najlepszy,
Grzechami Cię obrażamy,
Lecz Ty się zmiłuj nad nami
I~ześlij swe przebaczenie.
%
\item Bo chociaż bardzo niegodni,
Jesteśmy Twoim stworzeniem,
Więc wspomnij na to i~pomóż
Wysławiać życiem Twe imię.
%
\item Od zła nas wyzwól dawnego,
A~dobro pomnóż swą łaską,
I~uczyń Tobie miłymi
Już dziś i~w~całej wieczności.
%
\item Niech Tobie, w~Trójcy Jedyny,
Majestat będzie i~chwała,
A~Ty nam udziel owoców
Zbawczego czasu pokuty. Amen.\stophymnus

\starthourpart[title={Psalmodia}]

\ant[n=1][join] O~świcie~* napełnia nas, Panie, Twoja łaska.

\startpsalmus[title={Psalm 90}]
Panie, Ty dla nas byłeś ucieczką\pmed
z~pokolenia na pokolenie.\pfin
Zanim narodziły się góry,\pflx
nim powstał świat i~ziemia,\pmed
od wieku po wiek Ty jesteś Bogiem.\pfin
Obracasz w~proch człowieka,\pmed
i~mówisz: „Wracajcie, synowie ludzcy”.\pfin
Bo tysiąc lat w~Twoich oczach\pflx
jest jak wczorajszy dzień, który minął,\pmed
albo straż nocna.\pfin
Porywasz ich, stają się niby sen poranny,\pmed
jak trawa, która rośnie:\pfin
Rankiem zielona i~kwitnąca,\pmed
wieczorem więdnie i~usycha.\pfin
Zaprawdę, Twój gniew nas niszczy,\pmed
trwoży nas Twoje oburzenie.\pfin
Położyłeś przed sobą nasze grzechy,\pmed
nasze skryte winy w~świetle Twojego oblicza.\pfin
Wszystkie nasze dni mijają w~Twoim gniewie,\pmed
nasze lata dobiegają końca jak westchnienie.\pfin
Miarą naszego życia jest lat siedemdziesiąt,\pmed
osiemdziesiąt, gdy jesteśmy mocni.\pfin
A~większość z~nich, to trud i~marność,\pmed
bo szybko mijają, my zaś odlatujemy.\pfin
Któż może poznać siłę Twego gniewu\pmed
i~kto znieść zdoła moc Twego oburzenia?\pfin
Naucz nas liczyć dni nasze,\pmed
byśmy zdobyli mądrość serca.\pfin
Powróć, Panie, jak długo będziesz zwlekał?\pmed
Bądź litościwy dla sług Twoich.\pfin
Nasyć nas o~świcie swoją łaską,\pmed
abyśmy przez wszystkie dni nasze mogli się radować i~cieszyć.\pfin
Daj radość w~zamian za dni Twego ucisku,\pmed
za lata, w~których zaznaliśmy niedoli.\pfin
Niech sługom Twoim ukaże się Twe dzieło,\pmed
a~Twoja chwała nad ich synami.\pfin
Dobroć Pana, Boga naszego, niech będzie nad nami\pflx
i~wspieraj pracę rąk naszych,\pmed
dzieło rąk naszych wspieraj!\pfin
\stoppsalmus

\antr
\ant[n=2][join] Wszystkie krańce ziemi~* głoszą chwałę Pana.

\startpsalmus[title={Pieśń (Iz 42, 10-16)}]
Śpiewajcie Panu pieśń nową,\pmed
Jego chwała aż po krańce ziemi.\pfin
Niech sławi Go morze i~to, co je napełnia,\pmed
i~wyspy razem z~tymi, którzy tam mieszkają.\pfin
Niech woła pustynia i~miasta,\pmed
osiedla plemienia Kedar.\pfin
Mieszkańcy Sela niech wznoszą okrzyki,\pmed
ze szczytów gór niech wołają radośnie.\pfin
Niech Panu oddają chwałę,\pmed
Jego cześć niech głoszą na wyspach.\pfin
Jak mocarz Pan kroczy,\pmed
jak wojownik pobudza odwagę;\pfin
Rzuca hasło, okrzyk wydaje wojenny,\pmed
nieprzyjaciół męstwem przewyższa.\pfin
„Długo milczałem,\pmed
powstrzymywałem siebie w~spokoju,\pfin
Teraz jak rodząca zakrzyknę,\pmed
będę dyszał gniewem, aż tchu mi zabraknie.\pfin
Wypalę góry i~pagórki,\pmed
sprawię, że wyschnie cała ich zieleń,\pfin
Rzeki w~stawy przemienię,\pmed
i~osuszę jeziora.\pfin
Uczynię, że niewidomi pójdą po nieznanej drodze,\pmed
powiodę ich ścieżkami, których nie znają;\pfin
W~światło zamienię ich ciemności,\pmed
a~miejsca wyboiste w~równinę”.\pfin
\stoppsalmus

\antr

\ant[n=3] Chwalcie imię Pana,~* wy, co stoicie w~Jego domu.

\startpsalmus[title={Psalm 135, 1-12}]
Chwalcie imię Pana,\pmed
chwalcie, Pańscy słudzy,\pfin
Którzy stoicie w~domu Pana,\pmed
na dziedzińcach domu Boga naszego.\pfin
Chwalcie Pana, bo Pan jest dobry,\pmed
śpiewajcie Jego imieniu, bo jest łaskawy.\pfin
Pan bowiem wybrał sobie Jakuba,\pmed
na własność wyłączną wybrał Izraela.\pfin
Wiem, że Pan jest wielki,\pmed
że nasz Pan jest nad wszystkimi bogami.\pfin
Cokolwiek spodoba się Panu,\pflx
uczyni na niebie i~na ziemi,\pmed
na morzu i~na wszystkich głębinach.\pfin
Z~krańców ziemi sprowadza chmury,\pflx
wywołuje deszcz błyskawicami\pmed
i~dobywa wiatr ze swoich komór.\pfin
Poraził pierworodnych w~Egipcie,\pmed
wśród ludzi i~wśród zwierząt.\pfin
W~tobie, kraju egipski, zdziałał znaki i~cuda\pmed
przeciw faraonowi i~wszystkim jego sługom.\pfin
Poraził wiele narodów\pmed
i~zgładził potężnych królów:\pfin
Amoryckiego króla Sichona i~Oga, króla Baszanu,\pmed
i~wszystkich królów kananejskich,\pfin
A~ziemię ich dał w~posiadanie,\pmed
w~posiadanie Izraela, swego narodu.\pfin
\stoppsalmus

\antr

\starthourpart[title={Czytanie\hfill\tf Wj 4,21-23}]
Pan rzekł do Mojżesza: Gdy będziesz zbliżał się do Egiptu, pamiętaj o~władzy
czynienia wszelkich cudów, jaką ci dałem do ręki, i~okaż ją przed faraonem. Ja
zaś uczynię upartym jego serce, że nie zechce zezwolić na wyjście ludu. A~ty
wtedy powiesz do faraona: To mówi Pan: Synem moim pierworodnym jest Izrael.
Mówię ci: Wypuść mojego syna, aby mi cześć oddawał; bo jeśli zwlekać będziesz
z~wypuszczeniem go, to Ja ześlę śmierć na twego syna pierworodnego.

\starthourpart[title={Pieśń Zachariasza}]
\ant[title={Ant. do pieśni Zachariasza}]
Bóg jest duchem,~* potrzeba więc, by Jego czciciele~/ oddawali Mu cześć w~Duchu
i~prawdzie.
\startrubrica
Pieśń Zachariasza jak na \at{stronie}[benedictus].
\stoprubrica

\starthourpart[title={Prośby}]

Błogosławmy naszego Odkupiciela, który łaskawie wysłużył nam ten czas
zbawienia. Z~pokorą zanośmy do Niego nasze prośby:

\aklamacja Stwórz w~nas, o~Panie, nowego ducha.

\prosba Chryste, nasze życie, Ty zechciałeś, abyśmy przez chrzest zostali
pogrzebani z~Tobą w~śmierci i~z~Tobą zostali wskrzeszeni,
--- dopomóż nam dzisiaj postępować w~nowości życia.

\prosba Panie, Ty wszystkim dobrze czyniłeś,
--- spraw, abyśmy troszczyli się o~wspólne dobro wszystkich ludzi.

\prosba Daj, abyśmy zgodnie współpracowali w~doczesnej społeczności,
--- a~zarazem zdążali do wiecznej ojczyzny.

\prosba Chryste, lekarzu ciał i~dusz, ulecz rany naszego serca,
--- abyśmy stale korzystali z~Twej pomocy na drodze świętości.

Ojcze nasz.

\starthourpart[title={Modlitwa}]
Wszechmogący Boże, wejrzyj na zgromadzenie Twoich wiernych~\flxs i~spraw, aby
nasze dusze oczyszczone przez umartwienie ciała~\meds jaśniały pragnieniem
posiadania Ciebie. Przez naszego Pana Jezusa Chrystusa Twojego Syna,~\flxs
który z~Tobą żyje i~króluje w~jedności Ducha Świętego,~\meds Bóg, przez
wszystkie wieki wieków.

\starthourpart[title={Zakończenie \tf (piosenka dnia)}]

% Dzień  8: Obłok {{{1
\page[even]
\setupheader[state=high]
\setupfooter[state=high]
\externalfigure[d08-kadzidlo][width=\textwidth]
\startday[title={Obłok}]

\starthour[title={Wezwanie}] %{{{2
\input lib/panie-otworz

\ant Uwielbiajmy Pana,~* którego oblicze jaśnieje nad nami
\startrubrica
Psalm Wezwania nr FIXME \at{str.}[invit-ps095].
\stoprubrica

\input lib/zapalenie-B

\starthour[title={Jutrznia}] %{{{2

\starthourpart[title={Hymn}]
% http://brewiarz.pl/indeksy/pokaz.php3?id=3&nr=211
% Jutrznia: Wielki Post
% w oficjum dni powszednich
% LG tom II (Pallottinum 1984), s. 40
% Skrócona LG (Pallottinum 1991), s. 250
\starthymnus
\item Przed Tobą, Ojcze, stajemy skruszeni
I~obciążeni winami,
A~niosąc w~sobie bezsilną samotność,
Szukamy Twojej pomocy.
%
\item Bo przecież jesteś troskliwym pasterzem
I~nie chcesz śmierci grzesznika,
Lecz by się zwrócił do Ciebie w~pokorze,
A~przez to życie otrzymał.
%
\item W~tym świętym czasie powrotu do Ciebie,
Gdy dajesz łaskę pokuty,
O~miłosierdzie błagamy z~nadzieją
I~zło naprawić pragniemy.
%
\item Potężne światło, jaśniejsze od słońca,
Którego tarcza już wschodzi,
Wielbimy Ciebie, nasz Boże i~Panie,
W~Najświętszej Trójcy Jedyny. Amen.\stophymnus

\starthourpart[title={Psalmodia}]

\ant[n=1] Boże, Twoja droga jest święta,~* nikt nie dorówna wielkością
naszemu Bogu.

\startpsalmus[title={Psalm 77}]
Głos mój się wznosi do Boga, gdy wołam,\pmed
głos mój wznoszę do Boga, aby mnie usłyszał.\pfin
W~dniu mej niedoli szukam Pana,\pmed
w~nocy niestrudzenie wyciągam rękę.\pfin
Dusza moja jest niepocieszona,\pflx
jęczę, kiedy wspomnę Boga,\pmed
słabnie mój duch, gdy rozmyślam.\pfin
Ty spędzasz sen z~moich powiek,\pmed
z~niepokoju mówić nie potrafię.\pfin
Rozpamiętuję dni, które dawno minęły,\pmed
i~lata poprzednie wspominam.\pfin
Rozmyślam nocą w~swym sercu,\pmed
rozważam, a~duch mój docieka:\pfin
„Czy Bóg odrzuci na wieki\pmed
i~już nie będzie łaskawy?\pfin
Czy Jego łaskawość ustała na zawsze\pmed
i~słowo zamilkło na pokolenia?\pfin
Czy Bóg zapomniał o~litości,\pmed
czy w~gniewie powstrzymał swe miłosierdzie?”\pfin
I~mówię: „Jakże to bolesne,\pmed
że odwróciła się ode mnie prawica Najwyższego”.\pfin
Wspominam dzieła Pana,\pmed
oto wspominam Twoje dawne cuda.\pfin
Rozmyślam o~wszystkich Twych dziełach\pmed
i~czyny Twoje wspominam.\pfin
Boże, Twoja droga jest święta,\pmed
który z~bogów dorówna wielkością naszemu Bogu?\pfin
Ty jesteś Bogiem działającym cuda,\pmed
ludziom objawiłeś swą potęgę.\pfin
Ramieniem swoim Twój lud wybawiłeś,\pmed
synów Jakuba i~Józefa.\pfin
Boże, ujrzały Cię wody,\pflx
ujrzały Cię wody i~zadrżały,\pmed
wzburzyły się ich odmęty.\pfin
Chmury wylały wody,\pflx
zahuczały chmury\pmed
i~Twoje strzały się posypały.\pfin
Głos Twego grzmotu jak łoskot wozu,\pflx
pioruny świat rozjaśniły,\pmed
ziemia poruszyła się i~zatrzęsła.\pfin
Twoja droga wiodła przez wody,\pflx
Twoja ścieżka przez wodne obszary\pmed
i~nie znać było Twych śladów.\pfin
Wiodłeś Twój lud jak trzodę\pmed
ręką. Mojżesza i~Aarona.\pfin
\stoppsalmus

\antr

\ant[n=2] Moje serce raduje się w~Panu,~* który poniża i~wywyższa.

\startpsalmus[title={Pieśń (1 Sm 2, 1-10)}]
Moje serce raduje się w~Panu,\pmed
dzięki Niemu moc moja wzrasta.\pfin
Szeroko otwarłam usta przeciw moim wrogom,\pmed
bo cieszyć się mogę Twoją pomocą.\pfin
Nikt nie jest tak święty jak Ty, Panie,\pflx
poza Tobą bowiem nie ma nikogo,\pmed
prócz naszego Boga nie ma innej ostoi.\pfin
Nie powtarzajcie słów pełnych pychy,\pmed
niech mowa harda z~ust waszych nie wychodzi,\pfin
Gdyż Pan jest Bogiem wszechwiedzącym\pmed
i~On ocenia uczynki.\pfin
Łuk potężnych się łamie,\pmed
a~mocą przepasują się słabi.\pfin
Syci za chleb się najmują,\pmed
głodni zaś odpoczywają.\pfin
Niepłodna rodzi siedmioro,\pmed
a~matka wielu dzieci usycha.\pfin
Pan daje śmierć i~życie,\pmed
wtrąca do Otchłani i~z~niej wyprowadza.\pfin
Pan czyni ubogim lub bogatym,\pmed
poniża i~wywyższa.\pfin
Biedaka z~prochu podnosi,\pmed
z~błota dźwiga nędzarza,\pfin
By go wśród książąt posadzić\pmed
i~dać mu tron chwały.\pfin
Fundamenty ziemi należą do Pana\pmed
i~na nich świat On położył.\pfin
On strzeże kroków swoich wiernych,\pflx
grzesznicy zaś zginą w~ciemnościach,\pmed
bo nie własną siłą człowiek zwycięża.\pfin
Pan wniwecz opornych obraca\pmed
i~przeciw nim grzmi na niebiosach.\pfin
Pan sądzi krańce ziemi,\pflx
króla obdarza potęgą\pmed
i~wywyższa moc swego pomazańca.\pfin
\stoppsalmus

\antr

\ant[n=3][join] Pan króluje,~* wesel się, ziemio.~\cont

\startpsalmus[title={Psalm 97}]
Pan króluje, wesel się, ziemio,\pmed
\cont~radujcie się, liczne wyspy!\pfin
Obłok i~ciemność wokół Niego,\pmed
prawo i~sprawiedliwość podstawą Jego tronu.\pfin
Przed Jego obliczem idzie ogień\pmed
i~dokoła pożera nieprzyjaciół Jego.\pfin
Jego błyskawice wszechświat rozświetlają,\pmed
a~ziemia drży na ten widok.\pfin
Góry jak wosk topnieją przed obliczem Pana,\pmed
przed obliczem Władcy całej ziemi.\pfin
Jego sprawiedliwość rozgłaszają niebiosa\pmed
i~wszystkie ludy widzą Jego chwałę.\pfin
Niech zawstydzą się wszyscy, którzy czczą posągi\pflx
i~chlubią się bożkami.\pmed
Niech wszystkie bóstwa hołd Mu oddają!\pfin
Słyszy o~tym i~cieszy się Syjon,\pflx
radują się miasta Judy\pmed
z~Twoich wyroków, o~Panie.\pfin
Ponad całą ziemią Tyś bowiem wywyższony\pmed
i~nieskończenie wyższy od wszystkich bogów.\pfin
Pan tych miłuje, którzy zła nienawidzą,\pflx
On strzeże dusz świętych swoich,\pmed
wydziera je z~rąk grzeszników.\pfin
Światło wschodzi dla sprawiedliwego\pmed
i~radość dla ludzi prawego serca.\pfin
Weselcie się w~Panu, sprawiedliwi,\pmed
i~sławcie Jego święte imię.\pfin
\stoppsalmus

\antr

\starthourpart[title={Czytanie\hfill\tf Lb 9,15-16}]
W~dniu, kiedy ustawiono przybytek, okrył go wraz z~Namiotem Świadectwa obłok,
i~od wieczora aż do rana pozostawał nad przybytkiem na kształt ognia. I~tak
działo się zawsze: obłok okrywał go w~dzień, a~w~nocy ---~jakby blask ognia.

\starthourpart[title={Pieśń Zachariasza}]
\ant[title={Ant. do pieśni Zachariasza}]
Nigdy nie słyszano,~* aby ktoś przywrócił wzrok niewidomemu od urodzenia;~/
tylko Chrystus, Syn Boży, tego dokonał.
\startrubrica
Pieśń Zachariasza jak na \at{stronie}[benedictus].
\stoprubrica


\starthourpart[title={Prośby}]
Uwielbiajmy Boga nieskończonej dobroci i~błagajmy Go przez Jezusa Chrystusa,
który zawsze żyje, aby się wstawiać za nami:

\aklamacja Zapal w nas Panie, ogień swej miłości.

\prosba Spraw, miłosierny Boże, abyśmy pełnili dzisiaj liczne dzieła miłości
--- i~okazywali życzliwość wszystkim naszym bliźnim.

\prosba Ty w~czasie potopu uratowałeś Noego przez arkę,
--- wybaw katechumenów przez wodę chrztu świętego.

\prosba Daj, abyśmy nie poprzestawali na karmieniu się tylko chlebem,
--- ale szukali pożywienia w~słowie, które pochodzi z~ust Twoich.

\prosba Pomóż nam usuwać spośród nas wszelką niezgodę,
--- daj, abyśmy się radowali pokojem i~miłością.

Ojcze nasz.

\starthourpart[title={Modlitwa}]
Boże, nasz Ojcze, przez paschalne misterium swojego Syna dokonałeś naszego
odkupienia,~\flxs dlatego w~sakramentalnych znakach głosimy śmierć
i~zmartwychwstanie Chrystusa,~\meds spraw, abyśmy stale doznawali wzrostu Twojej
łaski. Przez naszego Pana Jezusa Chrystusa, Twojego Syna,~\flxs który z~Tobą
żyje i~króluje w~jedności Ducha Świętego,~\meds Bóg, przez wszystkie wieki
wieków.

\starthourpart[title={Zakończenie \tf (piosenka dnia)}]

% Dzień  9: Baranek {{{1
\page[even]
\setupheader[state=high]
\setupfooter[state=high]
\externalfigure[d09-baranek][width=\textwidth]
\startday[title={Baranek}]

\starthour[title={Wezwanie}] %{{{2
\input lib/panie-otworz

\ant Uwielbiajmy Chrystusa Króla,~* który dla nas został podwyższony na krzyżu.
\startrubrica
Psalm Wezwania nr FIXME \at{str.}[invit-ps095].
\stoprubrica

\input lib/zapalenie-B

\starthour[title={Jutrznia}] %{{{2

\starthourpart[title={Hymn}]
% http://brewiarz.pl/indeksy/pokaz.php3?id=3&nr=020
% Kompleta: Części stałe
% LG tom I, wydanie II (Pallottinum 2006), s. 929
% LG tom II (Pallottinum 1984), s. 1205
% LG tom III (Pallottinum 1987), s. 1015
% LG tom IV (Pallottinum 1988), s. 971
% LG tom dodatkowy (tzw. wakacyjny; Pallottinum 2002), s. 1041
% Skrócona LG (Pallottinum 1991), s. 1167
\starthymnus
\item Chryste, Tyś dniem pełnym blasku,
Rozpraszasz nocne ciemności,
W~Tobie początek jest światła
I~nim obdarzasz wybranych.
%
\item Teraz więc Ciebie prosimy,
Byś strzegł nas w~mrocznych godzinach,
W~swoim pokoju zachował
I~przyniósł ulgę znużonym.
%
\item Kiedy już sen nas ogarnie,
Niech serce czuwa przy Tobie;
Tych, którzy Ciebie miłują,
Osłaniaj swoją prawicą.
%
\item Wejrzyj, Obrońco nasz, Boże,
I~wrogie oddal zasadzki,
Kieruj Twoimi wiernymi,
Boś własną Krwią ich odkupił.
%
\item Chryste, łagodny nasz Królu,
Niech Ciebie z~Ojcem i~Duchem
Wszystko, co żyje, wysławia
Przez całą wieczność bez końca. Amen.\stophymnus

\starthourpart[title={Psalmodia}]

\ant[n=1] Pokornym i~skruszonym sercem~* Ty, Boże, nie gardzisz.

\startpsalmus[title={Psalm 51}]
Zmiłuj się nade mną, Boże, w~łaskawości swojej,\pmed
w~ogromie swej litości zgładź nieprawość moją.\pfin
Obmyj mnie zupełnie z~mojej winy\pmed
i~oczyść mnie z~grzechu mojego.\pfin
Uznaję bowiem nieprawość moją,\pmed
a~grzech mój jest zawsze przede mną.\pfin
Przeciwko Tobie samemu zgrzeszyłem\pmed
i~uczyniłem, co złe jest przed Tobą,\pfin
Abyś okazał się sprawiedliwy w~swym wyroku\pmed
i~prawy w~swoim sądzie.\pfin
Oto urodziłem się obciążony winą\pmed
i~jako grzesznika poczęła mnie matka.\pfin
A~Ty masz upodobanie w~ukrytej prawdzie,\pmed
naucz mnie tajemnic mądrości.\pfin
Pokrop mnie hizopem, a~stanę się czysty,\pmed
obmyj mnie, a~nad śnieg wybieleję.\pfin
Spraw, abym usłyszał radość i~wesele,\pmed
niech się radują kości, które skruszyłeś.\pfin
Odwróć swe oblicze od moich grzechów\pmed
i~zmaż wszystkie moje przewinienia.\pfin
Stwórz, Boże, we mnie serce czyste\pmed
i~odnów we mnie moc ducha.\pfin
Nie odrzucaj mnie od swego oblicza\pmed
i~nie odbieraj mi świętego ducha swego.\pfin
Przywróć mi radość Twojego zbawienia\pmed
i~wzmocnij mnie duchem ofiarnym\pfin
Będę nieprawych nauczał dróg Twoich\pmed
i~wrócą do Ciebie grzesznicy.\pfin
Uwolnij mnie, Boże, od kary za krew przelaną,\pflx
Boże, mój Zbawco,\pmed
niech sławi mój język sprawiedliwość Twoją.\pfin
Panie, otwórz wargi moje,\pmed
a~usta moje będą głosić Twoją chwałę.\pfin
Ofiarą bowiem Ty się nie radujesz\pmed
a~całopalenia, choćbym dał, nie przyjmiesz.\pfin
Boże, moją ofiarą jest duch skruszony,\pmed
pokornym i~skruszonym sercem Ty, Boże, nie gardzisz.\pfin
Panie, okaż Syjonowi łaskę w~Twej dobroci,\pmed
odbuduj mury Jeruzalem.\pfin
Wtedy przyjmiesz prawe ofiary: dary i~całopalenia,\pmed
wtedy składać będą cielce na Twoim ołtarzu.\pfin
\stoppsalmus

\antr

\ant[n=2] Gdy się gniewasz, Panie,~* wspomnij na swe miłosierdzie.

\startpsalmus[title={Pieśń (Ha 3, 2-4. 13a. 15-19)}]
Usłyszałem, Panie, Twoje orędzie,\pmed
zobaczyłem, Panie, Twoje dzieło.\pfin
Gdy czas nadejdzie, niech ono odżyje,\pflx
pozwól nam je poznać, gdy zbliży się pora,\pmed
w~gniewnym zapale wspomnij na swą litość!\pfin
Bóg przychodzi z~Temanu,\pmed
Święty z~góry Paran.\pfin
Jego majestat okrywa niebiosa,\pmed
a~Jego chwały pełna jest ziemia.\pfin
Jego wspaniałość podobna do światła,\pflx
z~Jego rąk tryskają promienie,\pmed
moc Jego w~nich jest ukryta.\pfin
Wyszedłeś, aby lud swój ocalić\pmed
i~wybawić Twego pomazańca.\pfin
Konie bezbożnika wdeptałeś w~morze,\pmed
w~kipiącą topiel wód ogromnych.\pfin
Usłyszałem, i~me serce struchlało,\pmed
na ten głos moje wargi zadrżały,\pfin
Moje kości przeniknęła trwoga,\pmed
zachwiały się moje kroki.\pfin
Jednak w~spokoju czekam na klęskę,\pmed
która spotka lud naszych gnębicieli.\pfin
Choć drzewo figowe nie rozwija pąków\pmed
i~winnice nie wydają plonów,\pfin
Chociaż zawiodły zbiory oliwek,\pmed
a~pola nie przynoszą żywności,\pfin
Choć stada owiec znikają z~owczarni\pmed
i~nie ma wołów w~zagrodach,\pfin
Ja się jednak rozraduję w~Panu\pmed
i~rozweselę w~Bogu, moim Zbawicielu.\pfin
Pan, który jest moją siłą,\pflx
uczyni me nogi jak nogi jelenia\pmed
i~na wyżyny mnie wyprowadzi.\pfin
\stoppsalmus

\antr

\ant[n=3][join] Chwal, Jeruzalem,~* Pana.~\cont

\startpsalmus[title={Psalm 147 B}]
Chwal, Jeruzalem, Pana,\pmed
\cont~wysławiaj twego Boga, Syjonie!\pfin
Umacnia bowiem zawory bram twoich\pmed
i~błogosławi synom twoim w~tobie.\pfin
Zapewnia pokój twoim granicom\pmed
i~wyborną pszenicą ciebie darzy.\pfin
Zsyła na ziemię swoje polecenia,\pmed
a~szybko mknie Jego słowo.\pfin
On prószy śniegiem jak wełną\pmed
i~szron jak popiół rozsypuje.\pfin
On grad rozrzuca jak okruchy chleba,\pmed
od Jego mrozu ścinają się wody.\pfin
Posyła słowo, i~lód topnieje,\pmed
powieje wiatrem, i~rzeki płyną.\pfin
Oznajmił swoje słowo Jakubowi,\pmed
Izraelowi ustawy swe i~wyroki\pfin
Nie uczynił tego dla innych narodów,\pmed
nie oznajmił im swoich wyroków.\pfin
\stoppsalmus

\antr

\starthourpart[title={Czytanie\hfill\tf Wj 12,21-23}]
Mojżesz zwołał wszystkich starszych Izraela i~rzekł do nich: Odłączcie i~weźcie
baranka dla waszych rodzin i~zabijcie jako paschę. Weźcie gałązkę hizopu
i~zanurzcie ją we krwi, która jest w~naczyniu, i~krwią z~naczynia skropcie próg
i~oba odrzwia. Aż do rana nie powinien nikt z~was wychodzić przed drzwi swego
domu. A~gdy Pan będzie przechodził, aby porazić Egipcjan, a~zobaczy krew na
progu i~na odrzwiach, to ominie Pan takie drzwi i~nie pozwoli Niszczycielowi
wejść do tych domów, aby was zabijał.

\starthourpart[title={Pieśń Zachariasza}]
\ant[title={Ant. do pieśni Zachariasza}]
Łazarz, przyjaciel nasz, zasnął,~* lecz idę, aby go obudzić.
\startrubrica
Pieśń Zachariasza jak na \at{stronie}[benedictus].
\stoprubrica

\starthourpart[title={Prośby}]
Błogosławmy naszego Odkupiciela, który łaskawie wysłużył nam ten czas
zbawienia. Z~pokorą zanośmy do Niego nasze prośby:

\aklamacja Stwórz w~nas, o~Panie, nowego ducha.

\prosba Chryste, nasze życie, Ty zechciałeś, abyśmy przez chrzest zostali
pogrzebani z~Tobą w~śmierci i~z~Tobą zostali wskrzeszeni,
--- dopomóż nam dzisiaj postępować w~nowości życia.

\prosba Panie, Ty wszystkim dobrze czyniłeś,
--- spraw, abyśmy troszczyli się o~wspólne dobro wszystkich ludzi.

\prosba Daj, abyśmy zgodnie współpracowali w~doczesnej społeczności,
--- a~zarazem zdążali do wiecznej ojczyzny.

\prosba Chryste, lekarzu ciał i~dusz, ulecz rany naszego serca,
--- abyśmy stale korzystali z~Twej pomocy na drodze świętości.

Ojcze nasz.

\starthourpart[title={Modlitwa}]
Prosimy Cię, Panie, nasz Boże,~\flxs udziel nam łaski, abyśmy gorliwie
naśladowali miłość Twojego Syna,~\meds który oddał własne życie za zbawienie
świata. Przez naszego Pana, Jezusa Chrystusa, Twojego Syna,~\flxs który z~Tobą
żyje i~króluje w~jedności Ducha Świętego,~\meds Bóg, przez wszystkie wieki
wieków.

\starthourpart[title={Zakończenie \tf (piosenka dnia)}]

% Dzień 10: Pascha {{{1
\page[even]
\setupheader[state=high]
\setupfooter[state=high]
\externalfigure[d10-paschal][width=\textwidth]
\startday[title={Pascha}]

\starthour[title={Wezwanie}] %{{{2
\input lib/panie-otworz

\ant Uwielbiajmy Chrystusa Syna Bożego,~* który nas odkupił krwią swoją
\startrubrica
Psalm Wezwania nr FIXME \at{str.}[invit-ps095].
\stoprubrica

\input lib/zapalenie-B

\starthour[title={Jutrznia}] %{{{2

\starthourpart[title={Hymn}]
% http://brewiarz.pl/indeksy/pokaz.php3?id=3&nr=240
% Jutrznia: Triduum Paschalne
% Wielka Sobota
% LG tom II (Pallottinum 1984), s. 388
% Skrócona LG (Pallottinum 1991), s. 423
\starthymnus
\item Zbawicielu wszystkich ludzi,
Przyjmij śpiewy pełne żalu;
Zmiłuj się nad nami, Panie,
Przebacz ufającym Tobie.
%
\item Pradawnego wroga moce
Zniweczyłeś swoim krzyżem,
Który stał się znakiem wiary
Na wyznawców Twoich czole.
%
\item Już na zawsze nas uwolnij
Z~panowania zła i~grzechu,
Aby szatan nie mógł szkodzić
Odkupionym krwią najświętszą.
%
\item Ty ze względu na nas, Jezu,
Zejść raczyłeś do Otchłani
I~dłużników gorzkiej śmierci
Obdarzyłeś życiem wiecznym.
%
\item Gdy na rozkaz Twój wszechmocny
Świat osiągnie kres istnienia,
Znowu przyjdziesz w~blasku chwały,
By nagrodzić sprawiedliwych.
%
\item Ciebie więc prosimy, Chryste,
Byś uleczył nasze rany,
Ty, co z~Ojcem Twym i~Duchem
Godzien jesteś uwielbienia. Amen.\stophymnus

\starthourpart[title={Psalmodia}]

\ant[n=1][join] Chwalcie Pana,~* bo Jego łaska na wieki.

\startpsalmus[title={Psalm 136\par I}]
Chwalcie Pana, bo jest dobry,\pmed
bo Jego łaska na wieki.\pfin
Chwalcie Boga nad bogami,\pmed
bo Jego łaska na wieki.\pfin
Chwalcie Pana nad panami,\pmed
bo Jego łaska na wieki.\pfin
On sam cudów wielkich dokonał,\pmed
bo Jego łaska na wieki.\pfin
On w~swej mądrości uczynił niebiosa,\pmed
bo Jego łaska na wieki.\pfin
On rozpostarł ziemię nad wodami,\pmed
bo Jego łaska na wieki.\pfin
On uczynił światła ogromne,\pmed
bo Jego łaska na wieki.\pfin
Słońce, by dniem władało,\pmed
bo Jego łaska na wieki.\pfin
Księżyc i~gwiazdy, by władały nocą,\pmed
bo Jego łaska na wieki\pfin
\stoppsalmus
\antr

\ant[n=2] Wielkie i~godne podziwu~* są dzieła Twoje,~/ Panie, Boże wszechmogący.

\startpsalmus[title={II}]
On Egipcjanom pobił pierworodnych,\pmed
bo Jego łaska na wieki.\pfin
I~wywiódł spośród nich Izraela,\pmed
bo Jego łaska na wieki.\pfin
Ręką potężną, wyciągniętym ramieniem,\pmed
bo Jego łaska na wieki.\pfin
Rozdzielił Morze Czerwone,\pmed
bo Jego łaska na wieki.\pfin
I~środkiem morza przeprowadził Izraela,\pmed
bo Jego łaska na wieki.\pfin
Faraona z~wojskiem strącił w~Morze Czerwone,\pmed
bo Jego łaska na wieki.\pfin
I~prowadził swój lud przez pustynię,\pmed
bo Jego łaska na wieki.\pfin
On pobił wielkich królów,\pmed
bo Jego łaska na wieki.\pfin
On uśmiercił królów potężnych,\pmed
bo Jego łaska na wieki.\pfin
Sichona, króla Amorytów,\pmed
bo Jego łaska na wieki.\pfin
I~Oga, króla Baszanu,\pmed
bo Jego łaska na wieki.\pfin
A~ziemię ich dał na własność,\pmed
bo Jego łaska na wieki.\pfin
Na własność swemu słudze Izraelowi,\pmed
bo Jego łaska na wieki\pfin
Pamiętał o~nas w~naszym poniżeniu,\pmed
bo Jego łaska na wieki.\pfin
I~uwolnił nas od wrogów,\pmed
bo Jego łaska na wieki.\pfin
On pokarm daje każdemu ciału,\pmed
bo Jego łaska na wieki.\pfin
Chwalcie Boga, niebiosa,\pmed
bo Jego łaska na wieki.\pfin
\stoppsalmus

\antr

\ant[n=3] Pan mówi~* Błogosławieństwem mój lud się nasyci.

\startpsalmus[title={Pieśń (Jr 31, 10-14)}]
Słuchajcie, narody, słowa Pańskiego,\pmed
głoście je na wyspach odległych i~mówcie:\pfin
„Ten, który rozproszył Izraela, znów go zgromadzi\pmed
i~będzie nad nim czuwał jak pasterz nad swym stadem”.\pfin
Pan bowiem uwolni Jakuba,\pmed
wybawi go z~ręki silniejszych od niego.\pfin
Przyjdą z~weselem na szczyt Syjonu\pflx
i~rozradują się błogosławieństwem Pana:\pmed
zbożem, winem, oliwą, owcami i~wołami.\pfin
Ich życie stanie się podobne do zroszonego ogrodu\pmed
i~już nigdy więcej sił im nie zabraknie.\pfin
Wtedy dziewica rozweseli się w~tańcu,\pmed
a~młodzieńcy i~starcy cieszyć się będą.\pmed
Smutek ich bowiem w~radość zamienię,\pmed
pocieszę i~rozweselę po ich troskach.\pfin
Tłuszczem z~ofiar obficie obdarzę kapłanów,\pmed
i~błogosławieństwem mój lud się nasyci.\pfin
\stoppsalmus

\antr

\starthourpart[title={Czytanie\hfill\tf Wj 12,11-14}]
Tak zaś spożywać będziecie baranka: Biodra wasze będą przepasane, sandały na
waszych nogach i~laska w~waszym ręku. Spożywać będziecie pośpiesznie, gdyż jest
to Pascha na cześć Pana. Tej nocy przejdę przez Egipt, zabiję wszystko
pierworodne w~ziemi egipskiej od człowieka aż do bydła i~odbędę sąd nad
wszystkimi bogami Egiptu ---~Ja, Pan. Krew będzie wam służyła do oznaczenia
domów, w~których będziecie przebywać. Gdy ujrzę krew, przejdę obok i~nie będzie
pośród was plagi niszczycielskiej, gdy będę karał ziemię egipską. Dzień ten
będzie dla was dniem pamiętnym i~obchodzić go będziecie jako święto dla
uczczenia Pana. Po wszystkie pokolenia ---~na zawsze w~tym dniu świętować
będziecie.

\starthourpart[title={Pieśń Zachariasza}]
\ant[title={Ant. do pieśni Zachariasza}]
Gorąco pragnąłem~* spożyć tę Paschę z~wami,~/ zanim będę cierpiał.
\startrubrica
Pieśń Zachariasza jak na \at{stronie}[benedictus].
\stoprubrica

\starthourpart[title={Prośby}]
Zanośmy pokorne błagania do Chrystusa, Wiecznego Kapłana, którego Ojciec
namaścił Duchem Świętym, aby więźniom głosił wyzwolenie:

\aklamacja Panie, zmiłuj się nad nami.

\prosba Panie, ty udałeś się do Jerozolimy, aby podjąć mękę i~wejść do chwały,
--- doprowadź swój Kościół do wiekuistej Paschy.

\prosba Chryste, gdy byłeś podwyższony na krzyżu, Twój bok został przebity
włócznią żołnierza,
--- ulecz nasze rany.

\prosba Panie, Twój krzyż stał się drzewem życia,
--- udziel jego owoców odrodzonym przez chrzest święty.

\prosba Chryste, Ty wisząc na drzewie krzyża, przebaczyłeś pokutującemu łotrowi,
--- odpuść nam, grzesznym, nasze winy.

Ojcze nasz.

\starthourpart[title={Modlitwa}]
Boże, Ty przez mękę Chrystusa, Twojego Syna a~naszego Pana, zniweczyłeś śmierć,
która wynikła z~grzechu pierworodnego i~jako dziedzictwo przeszła na wszystkie
pokolenia,~\flxs spraw, abyśmy wszczepieni w~Chrystusa i~uświęceni przez łaskę
nosili w~sobie podobieństwo do Niego,~\meds jak z~konieczności natury nosiliśmy
podobieństwo do Adama. Przez naszego Pana Jezusa Chrystusa, Twojego Syna,~\flxs
który z~Tobą żyje i~króluje w~jedności Ducha Świętego,~\meds Bóg, przez
wszystkie wieki wieków.

\starthourpart[title={Zakończenie \tf (piosenka dnia)}]

% Dzień 11: Zwycięstwo {{{1
\page[even]
\setupheader[state=high]
\setupfooter[state=high]
\externalfigure[d10-krzyz-stula][width=\textwidth]
\startday[title={Zwycięstwo}]
%\starthour[title={Jutrznia}] %{{{2
%
%\starthourpart[title={Hymn}]
%% http://brewiarz.pl/indeksy/pokaz.php3?id=3&nr=181
%% Jutrznia: Okres Wielkanocny
%% w oficjum dni powszednich
%% LG tom II (Pallottinum 1984), s. 427
%% Skrócona LG (Pallottinum 1991), s. 451
%\starthymnus
%
%\item Minęły cienie i~mroki nocy,
%Światło jaśnieje nowego życia,
%Bo Dobry Pasterz, zabity dla nas,
%Wyszedł promienny ze swego grobu.
%
%\item Przecierpiał mękę na drzewie krzyża,
%Obmył grzeszników we krwi i~wodzie,
%Co wypłynęła jak zdrój ożywczy
%Z~serca przez włócznię ugodzonego.
%
%\item Jak lud wybrany po przejściu morza
%Dotarł do ziemi obietnic Boga,
%Tak my, zrodzeni w~paschalnym źródle,
%Naszą ojczyznę znajdziemy w~niebie.
%
%\item Niech zmartwychwstanie Bożego Syna
%Będzie nadzieją dla wszystkich ludzi,
%Nastało bowiem królestwo łaski,
%W~którym Zbawiciel obdarza szczęściem.
%
%\item Radosnym hymnem sławimy, Panie,
%Twoje zwycięstwo nad złem i~śmiercią;
%Niech będzie chwała przez wszystkie wieki
%Ojcu i~Tobie w~jedności Ducha. Amen.\stophymnus

% Dzień 12: Manna {{{1
\page[even]
\setupheader[state=high]
\setupfooter[state=high]
\externalfigure[d12-chleb][width=\textwidth]
\startday[title={Manna}]

\starthour[title={Wezwanie}] %{{{2
\input lib/panie-otworz

\ant Alleluja, uwielbiajmy Chrystusa Pana,~* który wstępuje do nieba,~/
alleluja.

\startrubrica
Psalm Wezwania nr FIXME \at{str.}[invit-ps095].
\stoprubrica

\input lib/zapalenie-C

\starthour[title={Jutrznia}] %{{{2

\starthourpart[title={Hymn}]
% Jutrznia: Uroczystość Najświętszego Ciała i Krwi Chrystusa
% Skrócona LG (Pallottinum 1991), s. 670
\starthymnus
\item Z nieba zstąpiło słowo Ojca
Nie opuszczając Jego tronu;
Przyszło dokonać swego dzieła
Przed zapadnięciem zmroku życia.
%
\item Zanim przez ucznia był wydany
Wrogom, co śmierć mu zadać mieli,
Pan siebie jako pokarm życia
Dał Apostołom przy wieczerzy.
%
\item Dla nich pod dwoma postaciami
Zawarł swą Krew i swoje Ciało,
Aby się żywił człowiek ,
Także dwojaki w swej istocie.
%
\item Przez swe wcielenie jest nam bratem,
Naszym pokarmem w świętej uczcie,
Ceną okupu za swą mękę,
Naszą nagrodą, gdy króluje.
%
\item Zbawcza Ofiaro, która bramy
Nieba otwierasz swoim wiernym,
Daj nam wytrwałość i odwagę
Pośród zmagania z wrogą mocą.
%
\item Niechaj na wieki będzie chwała
Bogu jednemu w Trzech Osobach,
Który niech życie nieskończone
Da nam w ojczyźnie obiecanej. Amen\stophymnus

\starthourpart[title={Psalmodia}]

\ant[n=1] Wzbudź swą potęgę, Panie,~* i~przyjdź nam z~pomocą.

\startpsalmus[title={Psalm 80}]
Usłysz, Pasterzu Izraela,\pflx
Ty, który jak trzodę prowadzisz ród Józefa,\pmed
Ty, który zasiadasz nad cherubami!\pfin
Ukaż się przed Efraimem, Beniaminem i~Manassesem,\pmed
wzbudź swą potęgę i~przyjdź nam z~pomocą.\pfin
Odnów nas, Boże,\pflx
i~rozjaśnij nad nami swoje oblicze,\pmed
a~będziemy zbawieni.\pfin
Panie, Boże Zastępów, jak długo gniewać się będziesz\pmed
pomimo modłów Twojego ludu?\pfin
Nakarmiłeś go chlebem płaczu,\pmed
obficie napoiłeś łzami.\pfin
Uczyniłeś nas przyczyną zwady sąsiadów,\pmed
a~wrogowie nasi z~nas szydzą.\pfin
Odnów nas, Boże Zastępów,\pflx
i~rozjaśnij nad nami swoje oblicze,\pmed
a~będziemy zbawieni.\pfin
Przeniosłeś winorośl z~Egiptu\pmed
i~zasadziłeś ją wygnawszy pogan.\pfin
Przygotowałeś dla niej glebę,\pmed
a~ona zapuściła korzenie i~napełniła ziemię.\pfin
W~jej cieniu skryły się góry,\pmed
jej gałęzie okryły potężne cedry.\pfin
Rozpostarła swe pędy aż do Morza,\pmed
aż do Rzeki swoje latorośle.\pfin
Dlaczego zburzyłeś jej ogrodzenie\pmed
i~każdy przechodzień zrywa jej grona?\pfin
Niszczy ją dzik leśny\pmed
i~obgryzają polne zwierzęta.\pfin
Powróć, Boże Zastępów,\pmed
wejrzyj z~nieba, spójrz i~nawiedź tę winorośl.\pfin
Chroń to, co zasadziła Twoja prawica,\pmed
latorośl, którą umocniłeś dla siebie.\pfin
A~ci, którzy ją spalili i~wycięli,\pmed
niech zginą od grozy Twojego oblicza.\pfin
Wyciągnij rękę nad mężem Twej prawicy,\pfin
nad synem człowieczym, którego umocniłeś w~swej służbie.\pfin
Już więcej nie odwrócimy się od Ciebie,\pmed
daj nam nowe życie, a~będziemy Cię chwalili.\pfin
Odnów nas, Panie, Boże Zastępów,\pflx
i~rozjaśnij nad nami swoje oblicze,\pmed
a~będziemy zbawieni.\pfin
\stoppsalmus

\antr

\ant[n=2] Pan czynów wspaniałych dokonał~* i~cała ziemia niech o~tym się dowie.

\startpsalmus[title={Pieśń (Iz 12, 1-6)}]
Wysławiam Cię, Panie,\pmed
bo rozgniewałeś się na mnie,\pfin
Lecz Twój gniew się uśmierzył\pmed
i~dałeś mi pociechę.\pfin
Oto Bóg jest moim zbawieniem,\pmed
Jemu zaufam i~bać się nie będę.\pfin
Pan jest moją pieśnią i~mocą\pmed
i~On się stał moim zbawieniem.\pfin
Wy zaś z~weselem czerpać będziecie\pmed
wodę ze zdrojów zbawienia.\pfin
Jeszcze w~owym dniu powiecie:\pmed
Chwalcie Pana, wzywajcie Jego imienia!\pfin
Ukażcie narodom Jego dzieła,\pmed
przypominajcie, że Jego imię jest chwalebne.\pfin
Śpiewajcie Panu, bo czynów wspaniałych dokonał\pmed
i~cała ziemia niech o~tym się dowie.\pfin
Wznoś okrzyki i~wołaj radośnie, mieszkanko Syjonu,\pmed
bo wielki jest wśród ciebie Święty Izraela.\pfin
\stoppsalmus
\antr

\ant[n=3] Radośnie śpiewajcie Bogu,~* który jest naszą mocą.~\cont

\startpsalmus[title={Psalm 81}]
Radośnie śpiewajcie Bogu, który jest naszą mocą,\pmed
\cont~wykrzykujcie na cześć Boga Jakuba!\pfin
Zacznijcie śpiewać i~w~bęben uderzcie,\pmed
w~cytrę słodko dźwięczącą i~lirę.\pfin
Zadmijcie w~róg w~czas nowiu,\pmed
w~czas pełni, w~nasz dzień uroczysty.\pfin
Bo tak ustanowiono w~Izraelu\pmed
przykazania Boga Jakuba.\pfin
Ustanowił to prawo dla Józefa,\pmed
gdy wyruszał z~ziemi egipskiej.\pfin
Słyszę słowa nieznane:\pflx
„Uwolniłem od brzemienia jego barki,\pmed
jego ręce porzuciły kosze.\pfin
Wołałeś w~ucisku, a~Ja cię ocaliłem,\pflx
odpowiedziałem ci z~grzmiącej chmury,\pmed
doświadczyłem cię przy wodach Meriba.\pfin
Słuchaj, mój ludu, chcę cię napomnieć;\pmed
obyś Mnie posłuchał, Izraelu!\pfin
Nie będziesz miał obcego boga,\pmed
cudzemu bogu nie będziesz się kłaniał.\pfin
Jam jest Pan, Bóg twój,\pflx
który cię wywiódł z~ziemi egipskiej,\pmed
szeroko otwórz usta, abym je napełnił.\pfin
Lecz mój lud nie posłuchał mego głosu,\pmed
Izrael nie był Mi posłuszny.\pfin
Zostawiłem ich przeto ich twardym sercom,\pmed
niech postępują według swych zamysłów.\pfin
Gdyby mój lud Mnie posłuchał,\pmed
a~Izrael kroczył moimi drogami,\pfin
Natychmiast bym zgniótł ich wrogów\pmed
i~obrócił rękę na ich przeciwników.\pfin
Schlebialiby Panu ci, którzy Go nienawidzą,\pmed
a~kara ich trwałaby na wieki.\pfin
A~jego bym karmił wyborną pszenicą\pmed
i~sycił miodem z~opoki”.\pfin
\stoppsalmus
\antr

\starthourpart[title={Czytanie\hfill\tf Mdr 16,20}]
Lud swój żywiłeś pokarmem anielskim i~dałeś im bez ich wysiłków gotowy chleb
z~nieba, zdolny dać wszelką rozkosz i~wszelki smak zaspokoić.

\starthourpart[title={Pieśń Zachariasza}]
\ant[title={Ant. do pieśni Zachariasza}]
Wstępuję do Ojca mojego i~Ojca waszego,~* Boga mojego i~Boga waszego.
\startrubrica
Pieśń Zachariasza jak na \at{stronie}[benedictus].
\stoprubrica

\starthourpart[title={Prośby}]
Nasz Pan, Jezus Chrystus, wywyższony nad ziemię, wszystkich pociągnął
do siebie. Wzywajmy Go z~radością i~wołajmy:

\aklamacja Tyś Królem chwały, Jezu Chryste.

\prosba Panie Jezu Chryste, Królu chwały, Ty raz ofiarowałeś się za ludzkie
grzechy i~jako zwycięzca wstąpiłeś na prawicę Ojca,
--- doprowadź zbawionych do pełnej świętości.

\prosba Wieczny Kapłanie i~Pośredniku Nowego Przymierza, Ty zawsze żyjesz, aby
się wstawiać za nami,
--- zbaw lud, który do Ciebie zanosi błaganie.

\prosba Ty po swej męce ukazałeś się żywy i~przez czterdzieści dni objawiałeś
się uczniom,
--- umocnij dzisiaj naszą wiarę.

\prosba Ty w~dniu dzisiejszym obiecałeś Apostołom zesłać Ducha Świętego, aby
byli Twoimi świadkami aż po krańce ziemi,
--- mocą Ducha utwierdzaj nasze świadectwo.

Ojcze nasz.

\starthourpart[title={Modlitwa}]
Wszechmogący Boże, wierzymy, że nasz Zbawiciel zasiada w~chwale po Twojej
prawicy,~\flxs wysłuchaj nasze błagania~\meds i~spraw, abyśmy zgodnie z~Jego
obietnicą odczuwali, że pozostaje z~nami aż do skończenia świata. Przez naszego
Pana Jezusa Chrystusa, Twojego Syna,~\flxs który z~Tobą żyje i~króluje
w~jedności Ducha Świętego,~\meds Bóg, przez wszystkie wieki wieków.

\starthourpart[title={Zakończenie \tf (piosenka dnia)}]

% Dzień 13: Przymierze {{{1
\page[even]
\setupheader[state=high]
\setupfooter[state=high]
\externalfigure[d13][width=\textwidth]
\startday[title={Przymierze}]

\starthour[title={Godzina czytań}] %{{{2

\input lib/boze-wejrzyj

\starthourpart[title={Hymn}]
% https://brewiarz.pl/indeksy/pokaz.php3?id=3&nr=195
% I i II Np Zesłania Ducha Świętego
% a także Np Okresu Wielkanocnego po Wniebowstąpieniu
% LG tom II (Pallottinum 1984), s. 730, 793, 810
% Skrócona LG (Pallottinum 1991), s. 563, 601
\starthymnus
\item O~Stworzycielu, Duchu, przyjdź,
Nawiedź dusz wiernych Tobie krąg,
Niebieską łaskę zesłać racz
Sercom, co dziełem są Twych rąk.
%
\item Pocieszycielem jesteś zwan
I~Najwyższego Boga dar,
Tyś namaszczeniem naszych dusz,
Zdrój żywy, miłość, ognia żar.
%
\item Ty darzysz łaską siedemkroć,
Bo moc z~prawicy Boga masz,
Przez Ojca obiecany nam,
Mową wzbogacasz język nasz.
%
\item Światłem rozjaśnij naszą myśl,
W~serca nam miłość świętą wlej
I~wątłą słabość naszych ciał
Pokrzep stałością mocy Twej.
%
\item Nieprzyjaciela odpędź w~dal
I~Twym pokojem obdarz wraz.
Niech w~drodze za przewodem Twym
Miniemy zło, co kusi nas.
%
\item Daj nam przez Ciebie Ojca znać,
Daj, by i~Syn poznany był,
I~Ciebie, jedno tchnienie Dwóch,
Niech wyznajemy z~wszystkich sił. Amen.\stophymnus

\starthourpart[title={Psalmodia}]

% TODO czy tu jest [join]?
\ant[n=1] Pan jest naszym Bogiem,~* na wieki pamięta o~swoim przymierzu

\startpsalmus[title={Psalm 105(104), 1-22\par I}]
Sławcie Pana, wzywajcie jego imienia\pmed
głoście jego dzieła wśród narodów\pfin
Śpiewajcie i~grajcie mu psalmy\pmed
rozsławiajcie wszystkie Jego cuda\pfin
Szczyćcie się Jego świętym imieniem\pmed
Niech się weseli serce szukających Pana\pfin
Rozważcie o~Panu i~Jego potędze\pmed
Zawsze szukajcie Jego oblicza\pfin
Pamiętajcie o~cudach, które On uczynił\pmed
o~Jego znakach, o~wyrokach ust Jego\pfin
Potomkowie Abrahama, słudzy Jego\pmed
synowie Jakuba, Jego wybrańcy\pfin
On, Pan, jest naszym Bogiem\pmed
Jego wyroki obejmują świat cały\pfin
Na wieki On pamięta o~swoim przymierzu\pmed
obietnicy danej tysiącu pokoleń\pfin
Przymierzu, które zawarł z~Abrahamem\pmed
przysiędze danej Izaakowi\pfin
Ustanowił je dla Jakuba jako prawo\pmed
dla Izraela, jako wieczne przymierze\pfin
Gdy powiedział: „Dam tobie ziemię Kanaan\pmed
na waszą własność dziedziczną”\pfin
Kiedy ich było niewielu\pmed
nielicznych przybyszów w~tym kraju\pfin
Wędrujących od plemienia do plemienia\pmed
od królestwa do jeszcze innego ludu\pfin
Nikomu nie pozwolił ich uciskać\pmed
karał z~ich powodu królów\pfin
„Nie dotykajcie moich pomazańców\pmed
i~moim prorokom nie czyńcie krzywdy”\pfin
Potem głód przywołał na ziemię\pmed
i~odebrał im cały zapas chleba\pfin
Wysłał przed nimi męża:\pmed
Józefa, którego sprzedano w~niewolę\pfin
Kajdanami ścisnęli mu nogi\pmed
jego kark zakuto w~żelazo\pfin
Aż się spełniła jego przepowiednia\pmed
i~poświadczyło słowo Pana\pfin
Król posłał, aby go uwolnić\pmed
wyzwolił go władca ludów\pfin
Ustanowił go panem nad swoim domem\pmed
władcą całej swojej posiadłości\pfin
By według swej myśli pouczał dostojników\pmed
a~jego doradców uczył mądrości\pfin
\stoppsalmus

\antr

% TODO [join]?
\ant[n=2] Pan Bóg pamięta~* o~swym świętym słowie

\startpsalmus[title={II}]
Potem Izrael wkroczył do Egiptu\pmed
Jakub był gościem w~kraju Chama\pfin
Bóg swój naród bardzo rozmnożył\pmed
uczynił go mocniejszym od jego wrogów\pfin
Ich serce odmienił, aby znienawidzili lud Jego\pmed
i~wobec sług Jego postępowali zdradziecko\pfin
Posłał wtedy sługę swojego Mojżesza\pmed
i~Aarona wybranego przez siebie\pfin
Okazali wśród nich znaki\pmed
i~cuda w~krainie Chama\pfin
Zesłał ciemności i~mrok nastał\pmed
lecz oni buntowali się przeciw Jego słowom\pfin
W~krew zamienił ich wody\pmed
i~pozabijał ryby\pfin
Od żab zaroiła się ich ziemia\pmed
nawet w~komnatach królewskich\pfin
Rzekł i~pojawiło się robactwo\pmed
i~w~całym kraju ich komary\pfin
Zamiast deszczu grad zesłał\pmed
palący ogień na ich ziemię\pfin
Powalił ich winnice i~figi\pmed
i~drzewa połamał w~ich kraju\pfin
Rzekł i~nadciągnęła szarańcza\pmed
nieprzeliczone roje świerszczy\pfin
Pożarły całą trawę w~ich kraju\pmed
i~zjadły płody ich ziemi\pfin
Pobił wszystkich pierworodnych w~ich kraju\pmed
cały kwiat ich potęgi\pfin
A~lud swój wyprowadził ze srebrem i~złotem\pmed
i~nikt nie był słaby w~jego pokoleniach\pfin
Egipcjanie cieszyli się z~ich wyjścia\pmed
bo ogarnął ich lęk przed nimi\pfin
Chmurę rozpostarł jako przykrycie\pmed
i~ogień, by świecił wśród nocy\pfin
Prosili i~zesłał im przepiórki\pmed
nasycił ich chlebem z~nieba\pfin
Rozdarł skałę i~trysnęła woda\pmed
popłynęła pustynią jak rzeka\pfin
Pamiętał bowiem o~swym świętym słowie\pmed
danym Abrahamowi, swojemu słudze\pfin
I~wyprowadził swój lud wśród radości\pmed
z~weselem swoich wybranych\pfin
Darował im ziemie narodów\pmed
i~zawładnęli dorobkiem ludów\pfin
By strzegli jego przykazań\pmed
i~zachowali prawa\pfin
\stoppsalmus

\antr

% TODO [join]?
\ant[n=3] Śpiewajcie Panu pieśń dziękczynną,~* On gromadzi Izraela

\startpsalmus[title={Psalm 147(146-147)}]
Chwalcie Pana, bo dobrze jest śpiewać psalmy Bogu\pmed
słodko jest Go wysławiać\pfin
Pan buduje Jeruzalem\pmed
gromadzi rozproszonych z~Izraela\pfin
On leczy złamanych na duchu\pmed
i~przewiązuje ich rany\pfin
On liczy wszystkie gwiazdy\pmed
i~każdej imię nadaje\pfin
Nasz Pan jest wielki i~potężny\pmed
a~Jego mądrość niewypowiedziana\pfin
Pan dźwiga pokornych\pmed
karki grzeszników zgina do ziemi\pfin
Śpiewajcie Panu pieśń dziękczynną\pmed
Bogu naszemu grajcie na harfie\pfin
On niebo chmurami osłania\pmed
przygotowuje deszcz na ziemi\pfin
Wzgórza trawą pokrywa\pmed
i~ziołami, które służą ludziom\pfin
On bydłu daje pokarm\pmed
i~pisklętom kruka, które wołają do Niego\pfin
Nie kocha się w~sile rumaka\pmed
ani w~potędze męża\pfin
Upodobał sobie w~tych, którzy cześć Mu oddają\pmed
którzy ufają Jego dobroci\pfin
Chwal Jeruzalem Pana\pmed
wysławiaj Twego Boga Syjonie\pfin
Umacnia bowiem zawory bram twoich\pmed
i~błogosławi synom twoim w~tobie\pfin
Zapewnia pokój swoim granicom\pmed
i~wyborną pszenicą hojnie ciebie darzy\pfin
Zsyła na ziemię swoje polecenia\pmed
a~szybko mknie jego słowo\pfin
On prószy śniegiem jak wełną\pmed
i~szron rozsypuje jak popiół\pfin
On grad rozrzuca jak okruchy chleba\pmed
od Jego mrozu ścinają się wody\pfin
Posyła słowo i~lód topnieje\pmed
powieje wiatrem i~rzeki płyną\pfin
Oznajmił swoje słowo Jakubowi\pmed
Izraelowi ustawy swe i~wyroki\pfin
Nie uczynił tego dla innych narodów\pmed
nie oznajmił im swoich wyroków\pfin
\stoppsalmus

\antr

% TODO przerobić responsoria Wigilii na tabelkę
\startdialog
\sym{K.} Pan mówi: Oto ja jestem z~wami\crlf
\sym{W.} Przez wszystkie dni aż do skończenia świata.
\stopdialog

\starthourpart[title={I Czytanie}]

Z Księgi Wyjścia.

\starthourpart[title={Responsorium}]

\startdialog
\sym{K.} Już was nie nazywam sługami
\sym{W.} Ale przyjaciółmi, ponieważ poznaliście wszystko, co wam
uczyniłem. Przyjmijcie Ducha Świętego Pocieszyciela,
którego przyśle wam Ojciec, Alleluja
\sym{K.} Jesteście przyjaciółmi moimi, gdy czynicie, co wam nakazałem
\sym{W.} Przyjmijcie Ducha Świętego Pocieszyciela, którego przyśle
wam Ojciec, Alleluja
\stopdialog

\starthourpart[title={II Czytanie}]

Z~konstytucji dogmatycznej „Lumen Gentium”.

\starthourpart[title={Responsorium}]

\startdialog
\sym{K.} Różne są działania
\sym{W.} Lecz ten sam Bóg, sprawca wszystkiego we wszystkich.
Wszystkim zaś objawia się Duch dla wspólnego dobra, Alleluja
\sym{K.} Wy jesteście Ciałem Chrystusa i poszczególnymi członkami
\sym{W.} Wszystkim zaś objawia się Duch dla wspólnego dobra, Alleluja
\stopdialog

\starthourpart[title={Ewangelia}]

\starthourpart[title={Hymn Ciebie, Boże, chwalimy}]
\startlines\psalmusverse=0\relax
Ciebie, Boga, wysławiamy,
Tobie, Panu, wieczna chwała.\pfin
Ciebie, Ojca, niebios bramy,
Ciebie wielbi ziemia cała.\pfin
Tobie wszyscy Aniołowie,
Tobie Moce i~niebiosy,\pfin
Cheruby, Serafinowie
ślą wieczystej pieśni głosy:\pfin
Święty, Święty nad Świętymi
Bóg Zastępów, Król łaskawy,\pfin
Pełne niebo z~kręgiem ziemi
majestatu Twojej sławy.\pfin
Apostołów Tobie rzesza,
chór Proroków pełen chwały,\pfin
Tobie hołdy nieść pośpiesza
Męczenników orszak biały.\pfin
Ciebie poprzez okrąg ziemi
z~głębi serca, ile zdoła,\pfin
Głosy ludów zgodzonymi
wielbi święta pieśń Kościoła.\pfin
Niezmierzonej Ojca chwały,
Syna, Słowo wiekuiste,\pfin
Z~Duchem, wszechświat wielbi cały:
Królem chwały Tyś, o~Chryste!\pfin
Tyś Rodzica Syn z~wiek wieka.
By świat zbawić swoim zgonem,\pfin
Przyoblókłszy się w~człowieka,
nie wzgardziłeś Panny łonem.\pfin
Tyś pokruszył śmierci wrota,
starł jej oścień w~męki dobie\pfin
I~rajskiego kraj żywota
otworzyłeś wiernym sobie.\pfin
Po prawicy siedzisz Boga,
w~chwale Ojca, Syn Jedyny,\pfin
Lecz gdy zabrzmi trąba sroga,
przyjdziesz sądzić ludzkie czyny.\pfin
Prosim, słudzy łask niegodni,
wspomóż, obmyj grzech, co plami,\pfin
Gdyś odkupił nas od zbrodni
drogiej swojej Krwi strugami.\pfin
Ze świętymi w~blaskach mocy
wiecznej chwały zlej nam zdroje,\pfin
Zbaw, o~Panie, lud sierocy,
błogosław dziedzictwo swoje!\pfin
Rządź je, broń po wszystkie lata,
prowadź w~niebios błogie bramy.\pfin
My w~dzień każdy, Władco świata,
Imię Twoje wysławiamy.\pfin
Po wiek wieków nie ustanie
pieśń, co sławi Twoje czyny.\pfin
O, w~dniu onym racz nas, Panie,
od wszelakiej ustrzec winy.\pfin
Zjaw swą litość w~życiu całym
tym, co żebrzą Twej opieki;\pfin
W~Tobie, Panie, zaufałem,
nie zawstydzę się na wieki.\pfin\stoplines

\starthourpart[title={Modlitwa}]
\startdialog
\sym{K.} Módlmy się.
\stopdialog

% W oryginalnej modlitwie w brewiarzu jest "przez misterium dnia dzisiejszego",
% tutaj to jest poprawione wg PNOR. Reszta jest z brewiarza, nie z PNOR-a.
Boże, Ty przez misterium Zesłania Ducha Świętego uświęcasz swój Kościół
ogarniający wszystkie ludy i~narody,~\flxs ześlij dary Ducha Świętego na całą
ziemię~\meds i~dokonaj w~sercach wiernych tych cudów, które zdziałałeś
w~początkach głoszenia Ewangelii. Przez naszego Pana Jezusa Chrystusa, Twojego
Syna,~\flxs który z~Tobą żyje i~króluje w~jedności Ducha Świętego,~\meds Bóg,
przez wszystkie wieki wieków.

\startdialog
\sym{K.} Błogosławmy Panu
\sym{W.} Bogu niech będą dzięki
\stopdialog

% Dzień 14: Ziemia obiecana {{{1
\page[even]
\setupheader[state=high]
\setupfooter[state=high]
\externalfigure[d14-niepokalana][width=\textwidth]
\startday[title={Ziemia obiecana}]

\starthour[title={Wezwanie}] %{{{2
\input lib/panie-otworz

\ant Uwielbiajmy Pana,~* który otworzył nam bramy nieba
\startrubrica
Psalm Wezwania nr FIXME \at{str.}[invit-ps095].
\stoprubrica

\input lib/zapalenie-C

\starthour[title={Jutrznia}] %{{{2

\starthourpart[title={Hymn}]
% Jutrznia: 15 sierpnia (Wniebowzięcie Najświętszej Maryi Panny)
% Skrócona LG (Pallottinum 1991), s. 1416
\starthymnus
\item Słońca promienie jak szatą Cię stroją,
Głowę Twą wieńczy światło gwiazd dwunastu,
Księżyc swe blaski pod stopy Twe ściele,
Panno przeczysta!
%
\item Śmierć zwyciężyłaś i grzech, i Otchłanie,
Troszczysz się o nas siedząc przy Twym Synu;
Ciebie wyznaje potężną Królową
Niebo i ziemia.
%
\item Otocz opieką wyznawców Chrystusa,
Bożej owczarni przywróć zbuntowanych,
Tych, którzy trwają w śmiertelnych ciemnościach,
Wezwij do światła.
%
\item Pociesz płaczących, ubogich i chorych,
Skruchą przejętym uproś przebaczenie,
Przywróć nadzieję zbawienia pewnego
Wszystkim cierpiącym.
%
\item Bóg Cię, Maryjo, koroną obdarzył,
Dla nas uczynił Matką i Królową;
Jemu niech będzie podzięka i chwała
W Trójcy Najświętszej. Amen.\stophymnus

\starthourpart[title={Psalmodia}]

\ant[n=1][join] Szczęśliwi, którzy mieszkają~* w~domu Twoim, Panie.

\startpsalmus[title={Psalm 84}]
Jak miłe są Twoje przybytki,\pmed
Panie Zastępów!\pfin
Dusza moja stęskniona pragnie przedsionków Pańskich,\pmed
serce moje i~ciało radośnie wołają do Boga żywego.\pfin
Nawet wróbel znajduje swój dom, a~jaskółka gniazdo,\pflx
gdzie złoży swe pisklęta:\pmed
przy ołtarzach Twoich, Panie Zastępów, Królu mój i~Boże!\pfin
Szczęśliwi, którzy mieszkają w~domu Twoim, Panie,\pmed
nieustannie wielbiąc Ciebie.\pfin
Szczęśliwi, których moc jest w~Tobie,\pmed
którzy zachowują ufność w~swym sercu.\pfin
Przechodząc suchą doliną, w~źródła ją zamieniają,\pmed
a~wczesny deszcz błogosławieństwem ją okryje.\pfin
Mocy im będzie przybywać,\pmed
ujrzą na Syjonie Boga nad bogami\pfin
Panie Zastępów, usłysz modlitwę moją,\pmed
nakłoń ucho, Boże Jakuba.\pfin
Spójrz, Boże, tarczo nasza,\pmed
wejrzyj na twarz Twojego pomazańca.\pfin
Doprawdy, dzień jeden w~przybytkach Twoich\pmed
lepszy jest niż innych tysiące.\pfin
Wolę stać w~progu domu mojego Boga,\pmed
niż mieszkać w~namiotach grzeszników.\pfin
Bo Pan Bóg jest słońcem i~tarczą,\pflx
On hojnie darzy łaską i~chwałą,\pmed
nie odmawia dobrodziejstw żyjącym nienagannie.\pfin
Panie Zastępów,\pmed
szczęśliwy człowiek, który ufa Tobie!\pfin
\stoppsalmus
\antr

\ant[n=2] Chodźcie, wejdźmy na górę Pana,~* do świątyni naszego Boga.

\startpsalmus[title={Pieśń (Iz 2, 2-5)}]
Stanie się na końcu czasów,\pmed
że góra świątyni Pańskiej\pfin
Mocno osiądzie na górskich szczytach\pmed
i~ponad pagórki się wzniesie.\pfin
Nadciągną do niej wszystkie narody,\pmed
liczne plemiona pójdą wołając:\pfin
„Chodźcie, wejdźmy na górę Pana,\pmed
do świątyni Boga Jakuba!\pfin
Niech On nas pouczy o~swoich drogach,\pmed
byśmy kroczyli Jego ścieżkami,\pfin
Bo Prawo wyjdzie ze Syjonu\pmed
i~z~Jeruzalem słowo Pana”.\pfin
On będzie rozjemcą pomiędzy ludami,\pmed
osądzi sprawy rozlicznych narodów.\pfin
Wtedy swe miecze przekują na pługi,\pmed
a~włócznie swoje na sierpy.\pfin
Naród przeciw narodowi nie wzniesie już oręża\pmed
i~nie będą się więcej ćwiczyć do wojny.\pfin
Chodźcie, domu Jakuba,\pmed
postępujmy w~światłości Pana!\pfin
\stoppsalmus
\antr

\ant[n=3][join] Śpiewajcie Panu,~* sławcie Jego imię.

\startpsalmus[title={Psalm 96}]
Śpiewajcie Panu pieśń nową,\pmed
śpiewaj Panu, ziemio cała.\pfin
Śpiewajcie Panu, sławcie Jego imię,\pmed
każdego dnia głoście Jego zbawienie.\pfin
Głoście Jego chwałę wśród wszystkich narodów,\pmed
rozgłaszajcie cuda Jego pośród wszystkich ludów.\pfin
Wielki jest Pan, godzien wszelkiej chwały,\pmed
budzi trwogę najwyższą, większą niż inni bogowie.\pfin
Bo wszyscy bogowie pogan są tylko ułudą,\pmed
Pan zaś stworzył niebiosa.\pfin
Przed Nim kroczą majestat i~piękno,\pmed
a~potęga i~blask w~Jego przybytku.\pfin
Oddajcie Panu, rodziny narodów,\pmed
oddajcie Panu chwałę i~uznajcie Jego potęgę.\pfin
Oddajcie Panu chwałę należną Jego imieniu,\pmed
przynieście dary i~wejdźcie na Jego dziedzińce.\pfin
Uwielbiajcie Pana w~świętym przybytku;\pmed
zadrżyj, ziemio cała, przed Jego obliczem.\pfin
Głoście wśród ludów, że Pan jest królem,\pflx
On świat tak utwierdził, że się nie zachwieje,\pmed
będzie sprawiedliwie sądził ludy.\pfin
Niech się radują niebiosa i~ziemia weseli,\pmed
niech szumi morze i~wszystko, co je napełnia.\pfin
Niech się cieszą pola i~wszystko, co na nich rośnie,\pmed
niech wszystkie drzewa w~lasach wykrzykują z~radości.\pfin
Przed obliczem Pana, który już się zbliża,\pmed
który już się zbliża, by osądzić ziemię.\pfin
On będzie świat sądził sprawiedliwie,\pmed
a~ludy według swej prawdy.\pfin
\stoppsalmus
\antr

\starthourpart[title={Czytanie\hfill\tf Ap 11,19 i 12,1}]
Potem Świątynia Boga w~niebie się otwarła, i~Arka Jego Przymierza ukazała
się w~Jego Świątyni, a~nastąpiły błyskawice, głosy, gromy, trzęsienie ziemi
i~wielki grad. Potem wielki znak się ukazał na niebie: Niewiasta obleczona
w~słońce i~księżyc pod jej stopami, a~na jej głowie wieniec z~gwiazd dwunastu.

\starthourpart[title={Pieśń Zachariasza}]
\ant[title={Ant. do pieśni Zachariasza}]
Piękna i~pełna blasku jesteś, Maryjo,~* jak jutrzenka wznosisz się do nieba.
\startrubrica
Pieśń Zachariasza jak na \at{stronie}[benedictus].
\stoprubrica

\starthourpart[title={Prośby}]
Oddając cześć naszemu Zbawicielowi, który narodził się z~Maryi Dziewicy,
zanośmy do Niego pokorne błagania:

\aklamacja Niech Wniebowzięta wstawia się za nami

\prosba Jezu, Słowo odwieczne, Ty wybrałeś sobie Niepokalaną Maryję na
mieszkanie,
--- wybaw nas od zepsucia grzechu.

\prosba Nasz Odkupicielu, Ty sprawiłeś, że Maryja Dziewica stała się godnym dla
Ciebie mieszkaniem i~przybytkiem Ducha Świętego,
--- daj, abyśmy byli na wieki świątynią Twego Ducha.

\prosba Królu wieków, Ty przyjąłeś swą Matkę z~ciałem i~duszą do niebieskiej
chwały,
--- spraw, abyśmy zawsze myślą przebywali w~niebie.

\prosba Panie nieba i~ziemi, Ty zechciałeś, aby Maryja jako Królowa stała po
Twojej prawicy,
--- daj, abyśmy zasłużyli na udział w~Jej chwale.

Ojcze nasz.

\starthourpart[title={Modlitwa}]
Boże, Ty wejrzałeś na pokorę Najświętszej Maryi Panny i~wyniosłeś Ją do godności
Matki Twojego Jedynego Syna, [a~w~dniu dzisiejszym] uwieńczyłeś ją najwyższą
chwałą,~\flxs spraw, abyśmy zbawieni przez śmierć i~zmartwychwstanie Jezusa
Chrystusa,~\meds przez Jej prośby mogli wejść do wiecznej chwały. Przez naszego
Pana Jezusa Chrystusa, Twojego Syna,~\flxs który z~Tobą żyje i~króluje
w~jedności Ducha Świętego,~\meds Bóg, przez wszystkie wieki wieków.

\starthourpart[title={Zakończenie \tf (piosenka dnia)}]

% Dzień 15: Droga przez pustynię {{{1
\page[even]
\setupheader[state=high]
\setupfooter[state=high]
\externalfigure[d15][width=\textwidth]
\startday[title={Droga przez pustynię}]

\starthour[title={Wezwanie}] %{{{2
\input lib/panie-otworz

\ant Uwielbiajmy Pana,~* który kieruje narodami na ziemi
\startrubrica
Psalm Wezwania nr FIXME \at{str.}[invit-ps095].
\stoprubrica

\input lib/zapalenie-C

\starthour[title={Jutrznia}] %{{{2

\starthourpart[title={Hymn}]
% Jutrznia: piątek IV tygodnia
% Skrócona LG (Pallottinum 1991), s. 1139
\starthymnus
\item Po mrokach nocy nowy dzień nastaje,
Dzień pełen blasku słonecznych promieni;
Niech jego światło wzmocni w nas nadzieję
Życia bez winy.
%
\item Wśród złudy świata wciąż szukamy drogi,
Która prowadzi do Bożej jasności,
Więc Cię prosimy, miłosierny Panie,
Chroń nas od grzechu.
%
\item Po chwilach próby, pokus i cierpienia,
Szczerej pokuty i żalu za błędy
Powróci pokój, radość i pociecha
Twojej dobroci.
%
\item Ucieczko nasza, Boże wszechmogący:
Ojcze i Synu, i Duchu jedności,
Niech Tobie będzie cześć i uwielbienie
Teraz i zawsze. Amen.\stophymnus

\starthourpart[title={Psalmodia}]

\ant[n=1] Opowiedzieli nam nasi ojcowie o~potędze Pana~* i~o~cudach,
których On dokonał.~/ Alleluja.

\startpsalmus[title={Psalm 78, 1-39}]
Słuchaj, mój ludu, nauki mojej,\pmed
nakłońcie wasze uszy na słowa ust moich;\pfin
Do przypowieści otworzę me usta,\pmed
wyjawię tajemnice zamierzchłego czasu.\pfin
Tego, cośmy usłyszeli i~poznali\pflx
i~co nam opowiedzieli nasi ojcowie,\pmed
nie będziemy ukrywać przed ich synami.\pfin
Opowiemy przyszłemu pokoleniu\pflx
chwałę Pana i~Jego potęgę\pmed
i~cuda, których dokonał.\pfin
Albowiem nadał w~Jakubie przykazania\pmed
i~ustanowił Prawo w~Izraelu,\pfin
Aby to, co zlecił naszym ojcom,\pmed
przekazali swym synom.\pfin
O~tym ma wiedzieć przyszłe pokolenie,\pmed
synowie, którzy się narodzą,\pfin
Że mają pokładać nadzieję w~Bogu\pflx
i~nie zapominać dzieł Bożych,\pmed
lecz strzec Jego poleceń.\pfin
A~niech nie będą jak ich ojcowie\pmed
pokoleniem opornych buntowników,\pfin
Pokoleniem, którego serce jest niestałe,\pmed
a~duch nie dochowuje wierności Bogu.\pfin
Synowie Efraima, uzbrojeni w~łuki,\pmed
zostali rozproszeni w~dniu bitwy.\pfin
Nie zachowali przymierza z~Bogiem\pmed
i~nie chcieli postępować według Jego Prawa.\pfin
Zapomnieli o~Jego dziełach\pmed
i~o~cudach, które im okazał.\pfin
Na oczach ich ojców uczynił cuda,\pmed
w~ziemi egipskiej, na polach Soanu.\pfin
Morze rozdzielił, by ich przeprowadzić,\pmed
wody ustawił jak groble.\pfin
We dnie prowadził ich obłokiem,\pmed
a~przez całą noc blaskiem ognia.\pfin
Rozłupał skały w~pustyni\pmed
i~jak wodną głębiną obficie ich napoił.\pfin
I~wydobył ze skały strumienie,\pmed
sprawił, że wody spłynęły jak rzeki.\pfin
\stoppsalmus

\antr

\ant[n=2] Wysławiajcie Pana, Boga naszego,~* pokłon Mu oddajcie w~Jego świątyni.

\startpsalmus[title={Psalm 99}]
Pan jest Królem, drżą narody,\pmed
zasiada na tronie z~cherubów, a~ziemia się trzęsie.\pfin
Wielki jest Pan na Syjonie,\pmed
wywyższony ponad wszystkie ludy.\pfin
Niech wielbią imię Twoje, wielkie i~straszliwe,\pmed
ono jest święte.\pfin
I~króluje Potężny, który sprawiedliwość kocha:\pflx
Tyś ład ustanowił,\pmed
wymierzasz sprawiedliwość i~prawo w~Jakubie.\pfin
Wysławiajcie Pana, naszego Boga,\pflx
padajcie przed podnóżkiem stóp Jego,\pmed
bo On jest święty.\pfin
Wśród Jego kapłanów są Mojżesz i~Aaron\pflx
i~Samuel wśród tych, którzy wzywali Jego imienia,\pmed
wzywali Pana, a~On ich wysłuchał.\pfin
Przemawiał do nich w~słupie obłoku,\pmed
a~oni strzegli przykazań i~prawa, które im nadał.\pfin
Boże, nasz Panie, Tyś ich wysłuchał,\pmed
łaskę im okazałeś, lecz karałeś występki.\pfin
Wysławiajcie Pana, Boga naszego,\pflx
pokłon oddajcie Jego świętej górze,\pmed
bo Pan nasz i~Bóg jest święty.\pfin
\stoppsalmus

\antr

\ant[n=3] Ich głos się rozchodzi po całej ziemi,~* ich słowa aż po krańce
świata.

\startpsalmus[title={Psalm 19 A, 2-7}]
Niebiosa głoszą chwałę Boga,\pmed
dzieło rąk Jego obwieszcza nieboskłon.\pfin
Dzień opowiada dniowi,\pmed
noc nocy wiadomość przekazuje.\pfin
Nie są to słowa ani nie jest to mowa,\pmed
których by dźwięku nie usłyszano.\pfin
Ich głos się rozchodzi po całej ziemi,\pmed
ich słowa aż po krańce świata.\pfin
Tam słońcu namiot postawił,\pflx
a~ono jak oblubieniec wychodzi ze swej komnaty,\pmed
cieszy się jak siłacz ruszający do biegu.\pfin
Ono wschodzi na krańcu nieba\pflx
i~biegnie aż po drugi kraniec,\pmed
a~nic przed jego żarem się nie schroni.\pfin
\stoppsalmus

\antr

\starthourpart[title={Czytanie\hfill\tf Mt 5,14-16}]
Wy jesteście światłem świata. Nie może się ukryć miasto położone na górze. Nie
zapala się też światła i~nie stawia pod korcem, ale na świeczniku, aby świeciło
wszystkim, którzy są w~domu. Tak niech świeci wasze światło przed ludźmi, aby
widzieli wasze dobre uczynki i~chwalili Ojca waszego, który jest w~niebie.

\starthourpart[title={Pieśń Zachariasza}]
\ant[title={Ant. do pieśni Zachariasza}]
Najdostojniejsza Królowo świata,~* Maryjo, zawsze Dziewico,~/ Ty porodziłaś
Chrystusa, naszego Pana i~Zbawiciela.
\startrubrica
Pieśń Zachariasza jak na \at{stronie}[benedictus].
\stoprubrica

\starthourpart[title={Prośby}]
Oddając cześć naszemu Zbawicielowi, który narodził się z~Maryi Dziewicy,
zanośmy do Niego pokorne błagania:

\aklamacja Niech Twoja Matka wstawia się za nami.

\prosba Jezu, Słońce sprawiedliwości, Twoje przyjście poprzedziła Niepokalana
Dziewica, jak pełna blasku jutrzenka,
--- spraw, abyśmy zawsze żyli w~promieniach Twojej światłości.

\prosba Dozwól nam, Panie, naśladować Twoją Matkę, która obrała najlepszą
cząstkę,
--- spraw, abyśmy za Jej przykładem szukali pokarmu dającego życie wieczne.

\prosba Zbawicielu świata, Ty mocą swojego odkupienia zachowałeś Twoją Matkę od
wszelkiej zmazy grzechu,
--- zachowaj nas od skażenia grzechem.

\prosba Nasz Odkupicielu, Ty sprawiłeś, że Maryja Dziewica stała się godnym
Ciebie mieszkaniem i~przybytkiem Ducha Świętego,
--- daj, abyśmy byli na wieki świątynią Twego Ducha.

Ojcze nasz.

\starthourpart[title={Modlitwa}]
Boże, Ty miłujesz ludzi, pokornie Cię błagamy, ześlij na nas obfitą łaskę Ducha
Świętego~\flxs i~spraw, abyśmy postępując zgodnie z~naszym powołaniem, dawali
świadectwo Ewangelii~\meds i~dążyli z~ufnością i~w~pokoju do zjednoczenia
wszystkich wierzących. Przez naszego Pana Jezusa Chrystusa, Twojego Syna,~\flxs
który z~Tobą żyje i~króluje w~jedności Ducha Świętego,~\meds Bóg, przez
wszystkie wieki wieków.

\starthourpart[title={Zakończenie \tf (piosenka dnia)}]

%}}}1

% Dodatek {{{1

\page[odd]
\setupheader[state=high]
\setupfooter[state=high]
\startbooktitle[title={Dodatek}]
\dontleavehmode\vfill
\startalignment[middle]
\switchtobodyfont[36pt]\apropal Dodatek
\stopalignment
\vfill
\setupheadertexts[\setups{header2}]
\page[empty,odd]

\starttitle[title={Antyfony do pieśni Zachariasza}] %{{{1

\startrubrica
Antyfony do pieśni Zachariasza na Niedziele Zwykłe i~na Uroczystości
przypadające podczas wakacji.
\stoprubrica

\startsubject[title={12 niedziela zwykła}] %{{{2
\datesubtitle{19--25 czerwca}

\ant[title={Rok A:}] Nie bójcie się tych, którzy zabijają ciało~* lecz duszy
zabić nie mogą.

\ant[title={Rok B:}] Jezus wstał,~* rozkazał wichrowi i~rzekł do jeziora:~/
Milcz, ucisz się.~/ Wicher się uspokoił i~nastała głęboka cisza.

\ant[title={Rok C:}] Syn Człowieczy~* musi wiele wycierpieć,~/ będzie odrzucony
i~zabity,~/ a~trzeciego dnia zmartwychwstanie.

\startsubject[title={13 niedziela zwykła}] %{{{2
\datesubtitle{26 czerwca -- 2 lipca}

\ant[title={Rok A:}] Kto was przyjmuje,~* Mnie przyjmuje:~/ a~kto Mnie
przyjmuje, przyjmuje Tego, który mnie posłał.

\ant[title={Rok B:}] Jezus widząc chorą kobietę~* rzekł do niej:~/ Córko, twoja
wiara cię uzdrowiła,~/ idź w~pokoju.

\ant[title={Rok C:}] Zostaw umarłym grzebanie ich umarłych,~* a~ty idź i~głoś
królestwo Boże.

\startsubject[title={14 niedziela zwykła}] %{{{2
\datesubtitle{3--9 lipca}

\ant[title={Rok A:}] Weźcie moje jarzmo na siebie~* i~uczcie się ode Mnie, /~bo
jestem cichy i~pokorny sercem.

\ant[title={Rok B:}] Zaprawdę powiadam wam:~* Żaden prorok nie jest mile
widziany w~swojej ojczyźnie.

\ant[title={Rok C:}] Gdy wejdziecie do jakiego domu,~* najpierw mówcie: Pokój
temu domowi.~/ I~wasz pokój spocznie na nim.

\startsubject[title={15 niedziela zwykła}] %{{{2
\datesubtitle{10--16 lipca}

\ant[title={Rok A:}] Jezus powiedział swoim uczniom:~* Wam dano poznać tajemnice
królestwa niebieskiego,~/ innym zaś przez przypowieści.

\ant[title={Rok B:}] Uczniowie Jezusa wyszli~* i~wzywali do nawrócenia.

\ant[title={Rok C:}] Pewien Samarytanin~* będąc w~podróży przechodził obok
poranionego.~/ Gdy go zobaczył, wzruszył się głęboko~/ i~opatrzył mu rany.

\startsubject[title={16 niedziela zwykła}] %{{{2
\datesubtitle{17--23 lipca}

\ant[title={Rok A:}] Królestwo niebieskie podobne jest do zaczynu, * który
kobieta wzięła i~włożyła w~trzy miary mąki, / aż się wszystko zakwasiło.

\ant[title={Rok B:}] Pójdźcie sami osobno na miejsce pustynne * i~odpocznijcie
nieco.

\ant[title={Rok C:}] Maria usiadła u~nóg Pana * i~przysłuchiwała się Jego
słowom.

\startsubject[title={17 niedziela zwykła}] %{{{2
\datesubtitle{24--30 lipca}

\ant[title={Rok A:}] Królestwo niebieskie~* jest podobne do sieci zarzuconej
w~morze,~/ kiedy się napełniła, na brzeg ją wyciągnęli~/ i~wybrali dobre ryby,
a~złe odrzucili.

\ant[title={Rok B:}] Pięcioma chlebami~* i~dwiema rybami~/ Chrystus nasycił pięć
tysięcy ludzi.

\ant[title={Rok C:}] Proście, a~będzie wam dane,~* szukajcie, a~znajdziecie,~/
kołaczcie, a~będzie wam otworzone.

\startsubject[title={18 niedziela zwykła}] %{{{2
\datesubtitle{31 lipca -- 6 sierpnia}

\ant[title={Rok A:}] Pięcioma chlebami~* i~dwiema rybami~/ Chrystus nasycił pięć
tysięcy ludzi.

\ant[title={Rok B:}] Zaprawdę, zaprawdę powiadam wam:~* Nie Mojżesz dał wam
chleb z~nieba,~/ ale dopiero Ojciec mój da wam prawdziwy chleb z~nieba.

\ant[title={Rok C:}] Gromadźcie sobie skarby w~niebie,~* gdzie ani mól, ani rdza
nie niszczą.

\startsubject[title={19 niedziela zwykła}] %{{{2
\datesubtitle{7--14 sierpnia}

\ant[title={Rok A:}] Jezus powiedział do zatrwożonych uczniów:~* Odwagi, Ja
jestem, nie bójcie się.

\ant[title={Rok B:}] Zaprawdę, zaprawdę powiadam wam:~* Kto we Mnie wierzy, ma
życie wieczne.

\ant[title={Rok C:}] Szczęśliwi słudzy,~* których pan zastanie czuwających, gdy
nadejdzie.

\startsubject[title={20 niedziela zwykła}] %{{{2
\datesubtitle{14--20 sierpnia}

\ant[title={Rok A:}] Kobieta kananejska przyszła do Jezusa~* i~oddała Mu pokłon,
mówiąc:~/ Panie, dopomóż mi.

\ant[title={Rok B:}] Ciało moje jest prawdziwym pokarmem,~* a~Krew moja jest
prawdziwym napojem.~/ Kto spożywa moje Ciało i~pije Krew moją, ma życie wieczne.

\ant[title={Rok C:}] Chrzest mam przyjąć~* i~jakiej doznaję udręki, aż się to
stanie.

\startsubject[title={21 niedziela zwykła}] %{{{2
\datesubtitle{21--27 sierpnia}

\ant[title={Rok A:}] Ty jesteś Piotr~--- Skała,~* i~na tej skale zbuduję mój
Kościół.

\ant[title={Rok B:}] Nikt nie może przyjść do Mnie,~* jeżeli mu to nie zostało
dane przez Ojca.

\ant[title={Rok C:}] Wielu przyjdzie ze wschodu i~zachodu~* i~zasiądą
z~Abrahamem, Izaakiem i~Jakubem w~królestwie niebieskim.

\startsubject[title={22 niedziela zwykła}] %{{{2
\datesubtitle{28 sierpnia -- 3 września}

\ant[title={Rok A:}] Cóż za korzyść odniesie człowiek,~* choćby cały świat
zyskał,~/ a~na swej duszy szkodę poniósł?

\ant[title={Rok B:}] Przyjmijcie w~duchu łagodności~* zaszczepione w~was
słowo,~/ które ma moc zbawić dusze wasze.

\ant[title={Rok C:}] Każdy, kto się wywyższa,~* będzie poniżony,~/ a~kto się
poniża będzie wywyższony.

\startsubject[title={23 niedziela zwykła}] %{{{2
\datesubtitle{4--10 września}

\ant[title={Rok A:}] Jeśli dwaj z~was na ziemi~* zgodnie o~coś prosić będą,~/ to
wszystkiego im użyczy mój Ojciec, który jest w~niebie.

\ant[title={Rok B:}] Jezus, spojrzawszy w~niebo, westchnął~* i~rzekł do
głuchoniemego:~/ Effatha, to znaczy: Otwórz się.

\ant[title={Rok C:}] Kto nie nosi swego krzyża, a~idzie za Mną,~* nie może być
moim uczniem.

\startsubject[title={24 niedziela zwykła}] %{{{2
\datesubtitle{11-17 września}

\ant[title={Rok A:}] Pan ulitował się na dwoim sługą,~* uwolnił go i~dług mu
darował.

\ant[title={Rok B:}] Syn Człowieczy~* musi wiele cierpieć,~/ będzie odrzucony
przez starszych i~będzie zabity,~/ ale trzeciego dnia zmartwychwstanie.

\ant[title={Rok C:}] Radość powstaje u~aniołów Bożych~* z~jednego grzesznika,
który się nawraca.

\startsubject[title={25 niedziela zwykła}] %{{{2
\datesubtitle{18--24 września}

\ant[title={Rok A:}] Królestwo niebieskie podobne jest do gospodarza,~* który
wyszedł wczesnym rankiem,~/ aby nająć robotników do swojej winnicy.

\ant[title={Rok B:}] Kto Mnie przyjmuje,~* przyjmuje Tego, który mnie posłał.

\ant[title={Rok C:}] Kto w~drobnej rzeczy jest wierny,~* ten i~w~wielkiej będzie
wierny.

\startsubject[title={26 niedziela zwykła}] %{{{2
\datesubtitle{25 września -- 1 października}

\ant[title={Rok A:}] Zaprawdę powiadam wam:~* Celnicy i~nierządnice wchodzą
przed wami do królestwa niebieskiego,~/ ponieważ uwierzyli.

\ant[title={Rok B:}] Kto wam poda kubek wody do picia,~* dlatego, że należycie
do Chrystusa,~/ zaprawdę powiadam wam, nie utraci swojej nagrody.

\ant[title={Rok C:}] Wspomnij, synu~* że za życia otrzymałeś swoje dobra,~/
a~Łazarz przeciwnie, niedolę;~/ teraz ty męki cierpisz, a~on doznaje pociechy.

\startsubject[title={Najświętszej Trójcy}] %{{{2
\datesubtitle{Niedziela po Zesłaniu Ducha Świętego --- Uroczystość}

\ant Niech będzie błogosławiony Bóg,~* Jedyny w~Trójcy Świętej,~/ który
stworzył świat i~nim rządzi;~/ Jemu chwała teraz i~na wieki.

\startsubject[title={Najświętszego Ciała i~Krwi Chrystusa}] %{{{2
\datesubtitle{Czwartek po uroczystości Najświętszej Trójcy --- Uroczystość}

\ant Ja jestem chlebem żywym,~* który zstąpił z~nieba.~/ Jeśli kto ten chleb
spożywa,~/ będzie żył na wieki.

\startsubject[title={Najświętszego Serca Pana Jezusa}] %{{{2
\datesubtitle{Piątek po 2 niedzieli po Zesłaniu Ducha Świętego --- Uroczystość}

\ant Dzięki serdecznej litości~* Bóg nas nawiedził~/ i~wyzwolił lud swój.~/
Alleluja.

\startsubject[title={Narodzenia Świętego Jana Chrzciciela}] %{{{2
\datesubtitle{24 czerwca --- Uroczystość}

\ant Otworzyły się usta Zachariasza~* i~prorokował, mówiąc:~/ Błogosławiony
Pan, Bóg Izraela.

\startsubject[title={Świętych Apostołów Piotra i Pawła}] %{{{2
\datesubtitle{29 czerwca --- Uroczystość}

\ant Szymon Piotr powiedział do Jezusa:~* Panie, do kogo pójdziemy?~/ Ty masz słowa
życia wiecznego.~/ A~myśmy uwierzyli i~poznali, że Ty jesteś Synem Bożym.

\startsubject[title={Wniebowzięcia NMP}] %{{{2
\datesubtitle{15 sierpnia --- Uroczystość}

\ant Piękna i~pełna blasku jesteś, Maryjo,~* jak jutrzenka wznosisz się do nieba.

\startsubject[title={NMP Częstochowskiej}] %{{{2
\datesubtitle{26 sierpnia --- Uroczystość}

\ant Błogosławmy Boga niebios~* i~wysławiajmy Go wobec wszystkich żyjących,~/
bo przez Maryję Dziewicę okazał nam swoje miłosierdzie.

\starttitle[title={Celebracja szóstego dnia}] %{{{1

\startrubrica
Następujący hymn jest recytowany szóstego dnia podczas celebracji.
\stoprubrica

\startpsalmus[title={Hymn (Syr 51,1-12)}]
Wychwalać Cię będę, Panie, Królu,\pmed
i~wysławiać Ciebie, Boga, Zbawiciela mego.\pfin
Wychwalać chcę imię Twoje,\pmed
ponieważ podporą i~pomocnikiem stałeś się dla mnie.\pfin
Ochroniłeś ciało moje od zguby,\pflx
od sieci oszczerczego języka\pmed
i~od warg wypowiadających kłamstwo;\pfin
a~wobec przeciwników\pflx
stałeś się pomocnikiem i~wybawiłeś mię,\pmed
według wielkości miłosierdzia i~Twego imienia,\pfin
od pokąsania przez tych, co są gotowi mnie połknąć,\pmed
od ręki szukających mej duszy,\pfin
z~wielu utrapień, jakich doznałem,\pmed
od uduszenia w~ogniu, który mnie otacza,\pfin
i~ze środka ognia, który nie ja zapaliłem,\pmed
z~głębokich wnętrzności Szeolu,\pfin
od języka nieczystego i~od słowa kłamliwego,\pmed
od oszczerstwa języka przewrotnego wobec króla.\pfin
Dusza moja zbliżyła się aż do śmierci,\pmed
a~życie moje było blisko Szeolu, na dole.\pfin
Ze wszystkich stron otoczyli mnie i~nie znalazłem wspomożyciela,\pmed
rozglądałem się za pomocą od ludzi, ale nie przyszła.\pfin
Wówczas wspomniałem na miłosierdzie Twoje, Panie,\pmed
i~na dzieła Twoje, te od wieków~---\pfin
że wybawisz tych, którzy cierpliwie czekają na Ciebie,\pmed
i~wyzwalasz ich z~ręki nieprzyjaciół.\pfin
Podniosłem z~ziemi mój głos błagalny\pmed
i~prosiłem o~uwolnienie od śmierci.\pfin
Wzywałem Pana: Ojcem moim jesteś\pmed
i~mocarzem, który mnie wyzwoli.\pfin
Nie opuszczaj w~dniach udręki,\pmed
a~w~czasie przewagi pysznych ---~bez pomocy!\pfin
Wychwalać będę bez przerwy Twoje imię\pmed
i~opiewać je będę w~uwielbieniu.\pfin
I~prośba moja została wysłuchana.\pflx
Wybawiłeś mnie bowiem z~zagłady\pmed
i~wyrwałeś z~przygody złowrogiej.\pfin
Dlatego będę Cię wielbił i~wychwalał,\pmed
i~błogosławił imieniu Pańskiemu.\pfin
\stoppsalmus

\starttitle[title={Psalmy Wezwania}] %{{{1

\pagereference[invit-ps095]%
\input lib/invit-ps095\page
\input lib/invit-ps100\page
\input lib/invit-ps024\page
\input lib/invit-ps067\page

\starttitle[title={Melodie}] %{{{1

\startsubsubject[title={Psalm Wezwania}]
 \externalfigure[lilypond/invitatorium]

\startsubsubject[title={Hymny}]
\startsubsubsubject[title={Melodia 1}]
\externalfigure[lilypond/hymn-1]
\startsubsubsubject[title={Melodia 2}]
\externalfigure[lilypond/hymn-2]
\startsubsubsubject[title={Melodia 3}]
\externalfigure[lilypond/hymn-3]
\startsubsubsubject[title={Melodia 4}]
\externalfigure[lilypond/hymn-4]

\startsubsubject[title={Psalmy}]
\startsubsubsubject[title={Antyfona 1}]
\externalfigure[lilypond/psalmy-1]
\startsubsubsubject[title={Antyfona 2}]
\externalfigure[lilypond/psalmy-2]

\startsubsubject[title={Pieśń Zachariasza}]
\startsubsubsubject[title={Melodia 1}]
\externalfigure[lilypond/benedictus-1]
\startsubsubsubject[title={Melodia 2}]
\externalfigure[lilypond/benedictus-2]

\startsubsubject[title={Prośby}]
\externalfigure[lilypond/prosby]

\startsubsubject[title={Ojcze nasz}]
\externalfigure[lilypond/ojcze-nasz]

\starttitle[title={Spis rzeczy}] %{{{1

\placecontent

\starttitle[title={Pieśń Zachariasza}] %{{{1

\startsubsubject[title={W~przekładzie ks. Wojciecha Danielskiego}]
\input lib/benedictus-Danielski

\startsubsubject[title={W~przekładzie powszechnym},reference=benedictus]%
\input lib/benedictus

%}}}1

\stoptext

% vim: fdm=marker tw=80 ts=4 sts=4 sw=4 et
